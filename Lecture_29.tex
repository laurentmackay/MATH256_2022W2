\documentclass[11pt, dvipsnames, handout]{beamer}
\newtoggle{full}
\settoggle{full}{true}

\newtoggle{covered}
\settoggle{covered}{false}

\newtoggle{presentable}
\settoggle{presentable}{false}

\newtoggle{dualscreen}
\settoggle{dualscreen}{false}

\usepackage{pgfplots}
%\pgfplotsset{compat = newest}

\usepackage{pgfpages}

\setbeamertemplate{note page}{\pagecolor{yellow!5}\vfill \insertnote \vfill}
\usepackage{collect}
\definecollection{notes}
\newcounter{notestaken}

\usepackage{xpatch}

\usepackage{ulem}

\usepackage[framemethod=tikz]{mdframed}

\usepackage{scalerel}
\usepackage{calc}

%\usepackage{enumitem}
\setlength\fboxsep{.2em}

\usepackage{graphicx} % Allows including images
\usepackage{booktabs} % Allows the use of \toprule, \midrule and \bottomrule in tables

\xpatchcmd{\itemize}
  {\def\makelabel}
  {\setlength{\itemsep}{0.65 em}\def\makelabel}
  {}
  {}


\xpatchcmd{\beamer@enum@}
  {\def\makelabel}
  {\setlength{\itemsep}{0.65 em}\def\makelabel}
  {}
  {}


%\makeatletter
%\renewcommand{\itemize}[1][]{%
%  \beamer@ifempty{#1}{}{\def\beamer@defaultospec{#1}}%
%  \ifnum \@itemdepth >2\relax\@toodeep\else
%    \advance\@itemdepth\@ne
%    \beamer@computepref\@itemdepth% sets \beameritemnestingprefix
%    \usebeamerfont{itemize/enumerate \beameritemnestingprefix body}%
%    \usebeamercolor[fg]{itemize/enumerate \beameritemnestingprefix body}%
%    \usebeamertemplate{itemize/enumerate \beameritemnestingprefix body begin}%
%    \list
%      {\usebeamertemplate{itemize \beameritemnestingprefix item}}
%      {%
%        \setlength\topsep{1em}%NEW
%        \setlength\partopsep{1em}%NEW
%        \setlength\itemsep{1em}%NEW
%        \def\makelabel##1{%
%          {%
%            \hss\llap{{%
%                \usebeamerfont*{itemize \beameritemnestingprefix item}%
%                \usebeamercolor[fg]{itemize \beameritemnestingprefix item}##1}}%
%          }%
%        }%
%      }
%  \fi%
%  \beamer@cramped%
%  \raggedright%
%  \beamer@firstlineitemizeunskip%
%}
%
%
%
%
%
%\makeatother

%\setlist[beamer@enum@]{topsep=1 em}
%\let\origcheckmark\checkmark %screw you dingbat
%\let\checkmark\undefined %screw you dingbat
%\usepackage{dingbat} 
%\let\checkmark\origcheckmark %screw you dingbat






%\usepackage{fontawesome}

\usepackage{mathtools}
\usepackage{etoolbox, calculator}

\usepackage{xcolor}
\usepackage{tikz}
\usetikzlibrary{arrows.meta}
\usetikzlibrary{calc}
\usepackage[nomessages]{fp}
\usepackage{transparent}
\usepackage{accsupp}
%\usepackage{color, xcolor}

%colorblind-friendly palette
%\definecolor{dblue}{RGB}{51,34,136}
\definecolor{lblue}{RGB}{136,204,238}
%\definecolor{green}{RGB}{17,119,51}
\definecolor{tan}{RGB}{221,204,119}
%\definecolor{mauve}{RGB}{204,102,119}

\usepackage{tcolorbox}



\usepackage{xifthen}
\usepackage{nicefrac}
\usepackage{amsmath}
\usepackage{amsthm}
\usepackage{amssymb}
\theoremstyle{definition}
\newtheorem*{define}{Definition}
\newtheorem*{recall}{Recall}


\DeclareMathOperator{\tr}{tr}

\usepackage{multicol}
%\setlength{\columnsep}{1cm}

\usepackage{tablists, amsmath,vwcol, cancel, polynom}
\usetikzlibrary{shapes, patterns, decorations.shapes}
%\usepackage{tikzpeople}
\tikzstyle{vertex}=[shape=circle, minimum size=2mm, inner sep=0, fill]
\tikzstyle{opendot}=[shape=circle, minimum size=2mm, inner sep=0, fill=white, draw]

% common math quick commands
\newcommand{\nicedd}[2]{\nicefrac{\text{d}#1}{\text{d}#2}}
\newcommand{\dd}[2]{\dfrac{\text{d}#1}{\text{d}#2}}
\newcommand{\pd}[2]{\dfrac{\partial #1}{\partial#2}}
\renewcommand{\d}[1]{\text{d}#1}
\newcommand{\ddn}[3]{\dfrac{\text{d}^{#3}#1}{\text{d}#2^{#3}}}
\newcommand{\pdn}[3]{\dfrac{\partial^{#3}#1}{\partial#2^{#3}}}
\newcommand{\p}[0]{^{\prime}}
\newcommand{\pp}[0]{^{\prime\prime}}
\newcommand{\op}[2][\text{L}]{#1 \left[ #2 \right]}

\newcommand{\lap}[1]{\mathcal{L}\left\{#1\right\}}
\newcommand{\lapinv}[1]{\mathcal{L}^{-1}\left\{#1\right\}}
\newcommand{\lapint}[1]{\int_0^\infty e^{-st}#1dt}
\newcommand{\evalat}[2]{\Big|_{#1}^{#2}}

\newcommand{\paren}[1]{ \left( #1 \right)}

\newcommand{\haxis}[4][\normcolor]{\draw[#1, <->] (-#2,0)--(#3,0) node[right]{$#4$}; }


\newcommand{\axis}[4]{\draw[\normcolor, <->] (-#1,0)--(#2,0) 
node[right]{$x$};
\draw[help lines, <->] (0,-#3)--(0,#4) node[above]{$y$};}

\newcommand{\laxis}[6]{\draw[<->] (-#1,0)--(#2,0) 
node[right]{$#5$};
\draw[ <->] (0,-#3)--(0,#4) node[above]{$#6$};}
\newcommand{\xcoord}[2]{
	\draw (#1,.2)--(#1,-.2) node[below]{$#2$};}
\newcommand{\textnode}[3]{
	\draw (#1,#2) node[below]{$#3$};}
	
\newcommand{\nxcoord}[2]{
	\draw (#1,-.2)--(#1,.2) node[above]{$#2$};}
\newcommand{\ycoord}[2]{
	\draw (.2,#1)--(-.2,#1) node[left]{$#2$};}
\newcommand{\nycoord}[2]{
	\draw (-.2,#1)--(.2,#1) node[right]{$#2$};}
\newcommand{\dlim}{\displaystyle\lim}
\newcommand{\dlimx}[1]{\displaystyle\lim_{x \rightarrow #1}}
\newcommand{\stickfig}[2]{
	\draw (#1,#2) arc(-90:270:2mm);
	\draw (#1,#2)--(#1,#2-.5) (#1-.25,#2-.75)--(#1,#2-.5)--(#1+.25,#2-.75) (#1-.2,#2-.2)--(#1+.2,#2-.2);}	

%\newcounter{example}
%\setcounter{example}{1}
%\newcounter{preFrameExample}
%\AtBeginEnvironment{frame}{\setcounter{preFrameExample}{\value{example}}}
%\newcommand{\ex}[1]{
%	 \setcounter{example}{\value{preFrameExample}}
%	 \textcolor{green}{\small\fbox{Example \arabic{example}: #1}}\\[8pt]
%	\stepcounter{example}}
%\newcommand{\exans}[1]{
%	\SUBTRACT{\value{preFrameExample}}{1}{\n}
%	 \textcolor{green}{\small\fbox{Solution \n: #1}}\\[8pt]}
\mode<presentation> {

% The Beamer class comes with a number of default slide themes
% which change the colors and layouts of slides. Below this is a list
% of all the themes, uncomment each in turn to see what they look like.


\usetheme{CambridgeUS}
\usecolortheme[named=black]{structure}


\newcommand{\studentcolor}[0]{ForestGreen}
\newcommand{\normcolor}[0]{NavyBlue}
\newcommand{\alertcolor}{Red}

\setbeamercolor{normal text}{fg=\normcolor}
\setbeamercolor{frametitle}{fg=\normcolor}
\setbeamercolor{section in head/foot}{fg=Black, bg=Gray!20}
\setbeamercolor{subsection in head/foot}{fg=Green!70!Black, bg=Gray!10}
\setbeamercolor{alerted text}{fg=\alertcolor}
\setbeamerfont{alerted text}{series=\bf}
\setbeamertemplate{enumerate items}[default]
\setbeamercolor{enumerate item}{fg=\normcolor}

\setbeamertemplate{footline} % To remove the footer line in all slides uncomment this line
%\setbeamertemplate{footline}[page number] % To replace the footer line in all slides with a simple slide count uncomment this line

\setbeamertemplate{navigation symbols}{} % To remove the navigation symbols from the bottom of all slides uncomment this line
}

\newcommand{\alertbox}[1]{\tcbox[on line, colframe=\alertcolor, colback=White, left=2pt,right=2pt,top=2pt,bottom=2pt]{\usebeamercolor*{normal text}#1}}


\newcommand{\startstu}{\setbeamercolor{normal text}{fg=\studentcolor}\usebeamercolor*{normal text}\setbeamercolor{enumerate item}{fg=\studentcolor}\usebeamercolor*{enumerate item}}
\newcommand{\stopstu}{\setbeamercolor{normal text}{fg=\normcolor}\usebeamercolor*{normal text}\setbeamercolor{enumerate item}{fg=\normcolor}\usebeamercolor*{enumerate item}}

\newcommand{\takenote}[1]{ \begin{collect}{notes}{}{}{}{}  #1  \end{collect}  \addtocounter{notestaken}{1}} %\ifthenelse{\value{notestaken}>0}{\hrulefill\\}{}

\makeatletter
\newcommand{\cover}{\alt{\beamer@makecovered}{\beamer@fakeinvisible}}
\newcommand{\ucover}[1]{\iftoggle{full}{}{\beamer@endcovered}\stopstu#1\startstu\iftoggle{full}{}{\beamer@startcovered}}
\makeatother

\newcommand{\skippause}{ \addtocounter{beamerpauses}{-1}}
\newcommand{\blockpres}{ \skippause \pause }

\newcommand{\studentify}[1]{\startstu #1  \stopstu }
\newcommand{\student}[1]{\iftoggle{full}{ \pause  \studentify{#1} }{\iftoggle{covered}{\studentify{#1}}{\cover{  #1 }}}}
\newcommand{\cstudent}[1]{\student{\begin{center} #1 \end{center}}}
\newcommand{\fullonly}[1]{\iftoggle{full}{ #1}{}}
\newcommand{\presentonly}[1]{\iftoggle{presentable}{ #1}{}}

\usepackage{xparse}
\usepackage{xifthen}

% shortcuts for commonly-used presentation elements
%\NewDocumentCommand{\slide}{o m}
% {\IfValueTF{#1}{\begin{frame}[t]{#1}}{\begin{frame}[t]} #2 \end{frame}}

\newtoggle{iscovered}

\newcommand{\slide}[2][]{%
%\setcounter{notestaken}{0}
\takenote{#2} 
%\ifthenelse{\equal{#1}{}}{\begin{frame}[t]}{\begin{frame}[t]{#1}} #2 \ifthenelse{\value{notestaken}>0}{ \note{\includecollection{notes}}}{} \end{frame}%
\ifthenelse{\equal{#1}{}}{\begin{frame}[t]}{\begin{frame}[t]{#1}} #2 \iftoggle{covered}{\settoggle{iscovered}{true}}{\settoggle{iscovered}{false}}  \note{ \iftoggle{iscovered}{}{\settoggle{covered}{true}} #2 \iftoggle{iscovered}{}{\settoggle{covered}{false}} } \end{frame}%
%\setcounter{notestaken}{0}
}
\newcommand{\defn}[2][]{%
 \setcounter{listcounter}{0}%
\ifthenelse{\equal{#1}{}}{\begin{block}{Definition}}{\begin{block}{#1 :}}%
 #2 \vspace{0.25em} \ifthenelse{\value{listcounter}>0}{\skippause}{} \pause \end{block}%
}



\newcommand{\arr}[2]{\begin{array}{#1}#2\end{array}}
\newcommand{\mat}[2]{\left[\arr{#1}{#2}\right]}
\newcommand{\carray}[1]{\arr{c}{#1}}
\newcommand{\larray}[1]{\arr{l}{#1}}
\newcommand{\rarray}[1]{\arr{r}{#1}}
\newcommand{\colvec}[1]{\mat{c}{#1}}

\newcommand{\itmz}[1]{\addtocounter{listcounter}{1} \begin{itemize}#1 \end{itemize} }
\newcommand{\subitem}[1]{\addtocounter{listcounter}{1} \begin{itemize} \item #1 \end{itemize}}
%
\newcommand{\enum}[1]{\addtocounter{listcounter}{1} \begin{enumerate} #1  \end{enumerate}  }


\newcommand{\algnlbl}[1]{\begin{align}#1  \end{align}} 
\newcommand{\algn}[1]{\begin{align*}#1  \end{align*}} 
\newcommand{\lgn}[1]{ \action<+->{#1} }
\newcommand{\slgn}[1]{\iftoggle{full}{\action<+->{ \startstu #1 \stopstu}}{ \cover{ #1 } } \takenote{$#1$}}

\newcommand{\chckmrk}{\alert{\checkmark}}

\usepackage{pifont}
\newcommand{\xmark}{\alert{\text{\large \ding{55}}}}

\newcommand{\return}[0]{\raisebox{.5ex}{\rotatebox[origin=c]{180}{$\Lsh$}}}
\usepackage{pbox}
%\newcommand{\ex}[1]{\rotatebox[origin=c]{10}{\uline{ex}}:$\;$\pbox[t][][b]{0.9\linewidth}{#1}}
\newcommand{\ex}[1]{\uline{ex}:$\;$\pbox[t][][t]{0.9\linewidth}{#1}}
\newcommand{\eg}[1]{e.g.,$\;$\pbox[t][][t]{0.9\linewidth}{#1}}
\newcommand{\tikzplot}[8][]{%
\begin{tikzpicture}

\begin{scope}[]%
\clip(-#2,-#4) rectangle (#3,#5);%
#8%
\end{scope}%
\laxis{#2}{#3}{#4}{#5}{#6}{#7}%
#1
\end{tikzpicture}%
}


\newcommand{\cancelslide}[1]{%
\begingroup%
\setbeamertemplate{background canvas}{%
\begin{tikzpicture}[remember picture,overlay]%
\draw[line width=2pt,red!60!black] %
  (current page.north west) -- (current page.south east);%
\draw[line width=2pt,red!60!black] %
  (current page.south west) -- (current page.north east);%
\end{tikzpicture}}%
#1%
\endgroup%
}
\renewcommand{\CancelColor}{\color{red}}
\newcommand{\twocols}[3][0.5]{\begin{columns}\begin{column}{#1\textwidth}#2\end{column}\hspace{1em}\vrule{}\hspace{1em}\begin{column}{#1\textwidth}#3\end{column}\end{columns}}

\newcommand{\twomini}[5][1]{\calculatespace \begin{minipage}[t]{\columnwidth}\begin{minipage}[][#1\contentheight][t]{#2\columnwidth}#4\end{minipage}\hfill\begin{minipage}[][#1\contentheight][t]{#3\columnwidth}#5\end{minipage}\end{minipage}}

\newcommand{\threemini}[7][1]{\calculatespace \begin{minipage}[t]{\columnwidth}\begin{minipage}[][#1\contentheight][t]{#2\columnwidth}#5\end{minipage}\hfill\begin{minipage}[][#1\contentheight][t]{#4\columnwidth}#6\end{minipage}\hfill\begin{minipage}[][#1\contentheight][t]{#3\columnwidth}#7\end{minipage}\end{minipage}}


\newcounter{listcounter}
\setcounter{listcounter}{0}



\newif\ifsidebartheme
\sidebarthemetrue

\newdimen\contentheight
\newdimen\contentwidth
\newdimen\contentleft
\newdimen\contentbottom
\makeatletter
\newcommand*{\calculatespace}{%
\contentheight=\paperheight%
\ifx\beamer@frametitle\@empty%
    \setbox\@tempboxa=\box\voidb@x%
  \else%
    \setbox\@tempboxa=\vbox{%
      \vbox{}%
      {\parskip0pt\usebeamertemplate***{frametitle}}%
    }%
    \ifsidebartheme%
      \advance\contentheight by-1em%
    \fi%
  \fi%
\advance\contentheight by-\ht\@tempboxa%
\advance\contentheight by-\dp\@tempboxa%
\advance\contentheight by-\beamer@frametopskip%
\ifbeamer@plainframe%
\contentbottom=0pt%
\else%
\advance\contentheight by-\headheight%
\advance\contentheight by\headdp%
\advance\contentheight by-\footheight%
\advance\contentheight by4pt%
\contentbottom=\footheight%
\advance\contentbottom by-4pt%
\fi%
\contentwidth=\paperwidth%
\ifbeamer@plainframe%
\contentleft=0pt%
\else%
\advance\contentwidth by-\beamer@rightsidebar%
\advance\contentwidth by-\beamer@leftsidebar\relax%
\contentleft=\beamer@leftsidebar%
\fi%
}
\makeatother


\iftoggle{dualscreen}{\setbeameroption{show notes on second screen=right}}{}

\usepackage{gensymb}
\begin{document}

\section{Lecture 28}\subsection{The wave equation with various boundary conditions}

\slide[Generalizing D'Alembert's Solution]{
\vspace{-1em}
The solution method is agnostic to boundary conditions:
\vspace{-.5em}
\algn{y_{tt}&=c^2y_{xx} \quad  & a<x<b \\   y(x,0)&= f(x)& y_t(x,0)&=g(x)}
\vfill
\[\boxed{y(x,t)=A(x-ct)+B(x+ct)}\]
\vfill
\algn{A(z) &= \frac12 \left[ F(z) - \frac1c\int_a^z G(x) dx\right]& B(z) &= \frac12 \left[ F(z) + \frac1c\int_a^z G(x)dx \right]}
\vfill
\student{
\algn{y(x,t) &= \frac{ F(x-ct) +F(x+ct)}{2} +\frac{\int_a^{x+ct} G(s)ds -\int_a^{x-ct} G(s) ds}{2c} \\
&=\boxed{ \frac{ F(x-ct) +F(x+ct)}{2} +\frac{1}{2c} \int_{x-ct} ^{x+ct} G(s)ds }
}
}
}


\slide[Generalizing D'Alembert's Solution]{\vspace{-2em}
\algn{y_{tt}&=c^2y_{xx} \quad  & a<x<b \\   y(x,0)&= f(x)& y_t(x,0)&=g(x)}\vfill
\[\boxed{y(x,t) = \frac{ F(x-ct) +F(x+ct)}{2} +\frac{1}{2c} \int_{x-ct} ^{x+ct} G(s)ds }\]\vfill
Pick $F(z)$ and $G(z)$ to match the boundary conditions.\vfill
\itmz{\item $y(a,t)=y(b,t)=0$\\ \student{$\Rightarrow$ $F$ \& $G$ are \uline{odd} periodic extensions of $f$ and $g$}\vfill
\item $y_x(a,t)=y_x(b,t)=0$\\ \student{$\Rightarrow$ $F$ \& $G$ are \uline{even} periodic extensions of $f$ and $g$}
}
\vfill
}

\slide[Separation of variables with zero-derivative boundaries]{
\vspace{-2.5em}
\algn{y_{tt}&=c^2y_{xx} \quad  0<x<L & y_x(0,t)&=y_x(L,t)=0 \\   y(x,0)&= f(x)& y_t(x,0)&=g(x)}
\vfill
Perform \uline{even} periodic extensions of the initial conditions
\vfill
\student{\algn{f(x)&=\frac{a_0}{2} +\sum_{n=1}^\infty a_n \cos\left( \frac{n \pi}{L} x\right)\\
g(x)&=\frac{d_0}{2} +\sum_{n=1}^\infty d_n \cos\left( \frac{n \pi}{L} x\right)}}\vfill

\[y(x,t) = \frac{d_0}{2}t + \frac{a_0}{2}+ \sum_{n=1}^\infty \cos \left( \frac{n \pi}{L} x\right) \left[ d_n\frac{L}{n\pi c} \sin\left( \frac{n\pi c}{L} t \right) +  a_n\cos\left( \frac{n\pi c}{L} t \right) \right] \]
}


\slide{
\vspace{-.5em}Suppose you are given\vspace{-.5em}
\algn{y_{tt}&=c^2y_{xx} \quad  0<x<L & y_x(0,t)&=y_x(L,t)=0 \\   y(x,0)&= f(x)& y_t(x,0)&=g(x)}
Compare the two methods for computing the solution to this problem
\vfill
\student{Separation of variables
\enum{\item Find the even periodic extension of $f(x)$
\subitem{$a_n=$ an integral \item $a_0=$ another integral}
\item
\[y(x,t) = \frac{a_0}{2}+ \sum_{n=1}^\infty  a_n \cos \left( \frac{n \pi}{L} x\right) \cos\left( \frac{n\pi c}{L} t  \right)\]
 }\vfill
Infinite sums are too costly, in practice we only get approximate solutions.
}\vfill
\footnotesize\url{https://www.desmos.com/calculator/auxhik8lsh}
}

\slide{
\vspace{-.5em}Suppose you are given\vspace{-.5em}
\algn{y_{tt}&=c^2y_{xx} \quad  0<x<L & y_x(0,t)&=y_x(L,t)=0 \\   y(x,0)&= f(x)& y_t(x,0)&=g(x)}
Compare the two methods for computing the solution to this problem
\vfill
\student{D'Alembert
\enum{\item Define an even extension of $f(x)$ onto [-L,L]
\[f^{\rm even}(x) = \begin{cases} f(x)& 0\leq x<L \\  -f(-x)& -L\leq x<0 \end{cases}\]
\vfill
\item

\vspace{-2.5em} \algn{y(x,t) &= \frac12 \left[  f^{\rm even}(mod(x-ct+L, 2L)-L)\right. \\& \left.+   f^{\rm even}(mod(x+ct+L, 2L)-L)  \right]}}
Exact solution, no infinite sums or integrals required.

}
\vfill
\footnotesize\url{https://www.desmos.com/calculator/auxhik8lsh}
}

\slide[Other boundary conditions?]{
Suppose we have $y(0,t)=y_0\neq0$ and $y(L,t)=y_L\neq0$\vfill\student{
Proceed like with the heat equation
\[y_{\rm p}(x) = y_0 + \frac{y_L-y_0}{L}x \]
define $w(x,t)=y(x,t)-y_p(x)$
\algn{w_{tt}&=c^2w_{xx} \quad 0<x<L & w(0,t)&=w(L,t)=0 \\   w(x,0)&= f(x)-y_p(x)& y_t(x,0)&=g(x)}\vfill
Same strategy applies for non-zero derivatives.
}
}

\slide[Periodic extensions of functions defined for $a<x< b$]{
We have seen how to make odd/even periodic extensions of functions defined for $0<x< L$
\student{\itmz{\item Graphically \vfill \item Fourier Series with period $2L$\subitem{even=Fourier Cosine Series \item odd=Fourier Sine Series}}}
\vfill
How does this work for a function defined for $a<x\leq b$?\vfill
\student{\enum{\item Define a new coordinate $\tilde{x}=x-a$
\item Take a Fourier series in $\tilde{x}$ with $L=b-a$
}
\vfill
Hint: You need to do this for Assignment Q5c
}

\vfill
}

\slide[What happens as $a\to-\infty$ and $b\to\infty$ ]{\vspace{-2em}
\algn{y_{tt}&=c^2y_{xx} \quad  & -\infty<x<\infty \\   y(x,0)&= f(x)& y_t(x,0)&=g(x)}
\[\boxed{y(x,t) = \frac{ f(x-ct) +f(x+ct)}{2} +\frac{1}{2c} \int_{x-ct} ^{x+ct} g(s)ds }\]\vfill
The wave extends over all space...\student{eventually.\vfill However, the wave travels at finite speed.\vfill
Information about the initial condition at a point $x_0$  is always contained in the interval  $[x_0-ct,x_0+ct]$.}

}

\slide[What type of solutions do we get for $-\infty<x<\infty$?]{
For $0<x<L$, we got 
\student{\algn{y(x,t) &\approx \sum_{n=1}^{\infty} \left[ a_n \cos\left(\frac{2\pi t}{T_n} \right)+ b_n\sin\left( \frac{2\pi  t}{T_n} \right)\right]\left[\sin\left(\frac{2\pi x}{\lambda_n} \right)+ \cos\left( \frac{2\pi x}{\lambda_n} \right)\right]\intertext{$a_n$ and $b_n$ are found from periodic extensions of the initial conditions}
T_n&=\frac{2L}{cn} \qquad \lambda_n=\frac{2L}{n}\qquad  \text{(periods of the vibrational modes)}\intertext{alterntively, we can rewrite this as}
y(x,t) &=\sum_{n=-\infty}^{\infty}  c_n e^{i \frac{2\pi }{T_n}t }e^{i \frac{2\pi }{\lambda_n}x }   = \sum_{n=-\infty}^{\infty} c_n e^{2\pi i\left(  \frac{x }{\lambda_n} + \frac{t }{T_n}\right)}\\
c_n&=\begin{cases} a_0/2 & n=0\\ \frac12 (a_n+b_n) & n>0 \\ \frac12 (a_{|n|}-b_{|n|}) & n<0 \end{cases}
}
}
}

\slide[What type of solutions do we get for $-\infty<x<\infty$?]{
For $-\infty < x < \infty $, we get \vfill
\student{
Inspired by the finite case, we try separation of variables with $y(x,t)=e^{\alpha x} e^{\beta t}$
\algn{y_{tt}&=c^2y_{xx}\\
\beta^2 e^{\alpha x} e^{\beta t} &= c^2 \alpha^2 e^{\alpha x} e^{\beta t}\\
  \beta^2 &= c^2 \alpha^2 &\beta&=\pm c \alpha
 }
\hrule\vspace{.25em}
For general initial conditions, $\beta$ and $\alpha$ are purely imaginary. Let $\alpha = i k$
\algn{y(x,t) = \int_{-\infty}^\infty w(k) e^{2\pi i k  (x+ct) }  dk}
\centerline{The function $w(k)$ is found  from a Fourier Transform of the initial conditions}
\centerline{NOT ON EXAM}\vspace{.25em}
\hrule
}

}

\slide[What type of solutions do we get for $-\infty<x<\infty$?]{
For $-\infty < x < \infty $, we get \vfill
\student{
For some very special initial conditions, Fourier Transforms are not necessary.
\vfill
\ex{$y_{tt}=c^2y_{xx}$ with $y(x,0)=e^{-2x}$}\vfill
Try $y_{1,2}(x,t)=e^{\alpha x \pm c \alpha t}$
\algn{\text{Initial condition: } y(x,0)=e^{-2x}&=e^{\alpha x} \Rightarrow \alpha =-2 \\
y_{1,2}(x,t)&= e^{-2 x \mp 2 c t}}
\vfill
}

}

\slide[Summary]{
\itmz{\item D'Alembert solution to the wave equation is complementary to the Fourier series solution:
\subitem{Same solution, different perspective}\vfill
 \item Two waves one moving left and another moving right \subitem{ The two waves are fully determined by the initial conditions. 
\item The initial conditions propagate through the domain at finite speed}\vfill
 \item Finite Domain: use appropriate periodic extension + wave interference to match boundary conditions. \vfill
\item Infinite Domain: No need to use periodic extensions, there are no boundary conditions.
}
}

\end{document}