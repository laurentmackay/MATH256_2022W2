\documentclass[11pt, dvipsnames, handout]{beamer}
\newtoggle{full}
\settoggle{full}{true}

\newtoggle{covered}
\settoggle{covered}{false}

\newtoggle{presentable}
\settoggle{presentable}{false}

\newtoggle{dualscreen}
\settoggle{dualscreen}{false}

\usepackage{pgfplots}
%\pgfplotsset{compat = newest}

\usepackage{pgfpages}

\setbeamertemplate{note page}{\pagecolor{yellow!5}\vfill \insertnote \vfill}
\usepackage{collect}
\definecollection{notes}
\newcounter{notestaken}

\usepackage{xpatch}

\usepackage{ulem}

\usepackage[framemethod=tikz]{mdframed}

\usepackage{scalerel}
\usepackage{calc}

%\usepackage{enumitem}
\setlength\fboxsep{.2em}

\usepackage{graphicx} % Allows including images
\usepackage{booktabs} % Allows the use of \toprule, \midrule and \bottomrule in tables

\xpatchcmd{\itemize}
  {\def\makelabel}
  {\setlength{\itemsep}{0.65 em}\def\makelabel}
  {}
  {}


\xpatchcmd{\beamer@enum@}
  {\def\makelabel}
  {\setlength{\itemsep}{0.65 em}\def\makelabel}
  {}
  {}


%\makeatletter
%\renewcommand{\itemize}[1][]{%
%  \beamer@ifempty{#1}{}{\def\beamer@defaultospec{#1}}%
%  \ifnum \@itemdepth >2\relax\@toodeep\else
%    \advance\@itemdepth\@ne
%    \beamer@computepref\@itemdepth% sets \beameritemnestingprefix
%    \usebeamerfont{itemize/enumerate \beameritemnestingprefix body}%
%    \usebeamercolor[fg]{itemize/enumerate \beameritemnestingprefix body}%
%    \usebeamertemplate{itemize/enumerate \beameritemnestingprefix body begin}%
%    \list
%      {\usebeamertemplate{itemize \beameritemnestingprefix item}}
%      {%
%        \setlength\topsep{1em}%NEW
%        \setlength\partopsep{1em}%NEW
%        \setlength\itemsep{1em}%NEW
%        \def\makelabel##1{%
%          {%
%            \hss\llap{{%
%                \usebeamerfont*{itemize \beameritemnestingprefix item}%
%                \usebeamercolor[fg]{itemize \beameritemnestingprefix item}##1}}%
%          }%
%        }%
%      }
%  \fi%
%  \beamer@cramped%
%  \raggedright%
%  \beamer@firstlineitemizeunskip%
%}
%
%
%
%
%
%\makeatother

%\setlist[beamer@enum@]{topsep=1 em}
%\let\origcheckmark\checkmark %screw you dingbat
%\let\checkmark\undefined %screw you dingbat
%\usepackage{dingbat} 
%\let\checkmark\origcheckmark %screw you dingbat






%\usepackage{fontawesome}

\usepackage{mathtools}
\usepackage{etoolbox, calculator}

\usepackage{xcolor}
\usepackage{tikz}
\usetikzlibrary{arrows.meta}
\usetikzlibrary{calc}
\usepackage[nomessages]{fp}
\usepackage{transparent}
\usepackage{accsupp}
%\usepackage{color, xcolor}

%colorblind-friendly palette
%\definecolor{dblue}{RGB}{51,34,136}
\definecolor{lblue}{RGB}{136,204,238}
%\definecolor{green}{RGB}{17,119,51}
\definecolor{tan}{RGB}{221,204,119}
%\definecolor{mauve}{RGB}{204,102,119}

\usepackage{tcolorbox}



\usepackage{xifthen}
\usepackage{nicefrac}
\usepackage{amsmath}
\usepackage{amsthm}
\usepackage{amssymb}
\theoremstyle{definition}
\newtheorem*{define}{Definition}
\newtheorem*{recall}{Recall}


\DeclareMathOperator{\tr}{tr}

\usepackage{multicol}
%\setlength{\columnsep}{1cm}

\usepackage{tablists, amsmath,vwcol, cancel, polynom}
\usetikzlibrary{shapes, patterns, decorations.shapes}
%\usepackage{tikzpeople}
\tikzstyle{vertex}=[shape=circle, minimum size=2mm, inner sep=0, fill]
\tikzstyle{opendot}=[shape=circle, minimum size=2mm, inner sep=0, fill=white, draw]

% common math quick commands
\newcommand{\nicedd}[2]{\nicefrac{\text{d}#1}{\text{d}#2}}
\newcommand{\dd}[2]{\dfrac{\text{d}#1}{\text{d}#2}}
\newcommand{\pd}[2]{\dfrac{\partial #1}{\partial#2}}
\renewcommand{\d}[1]{\text{d}#1}
\newcommand{\ddn}[3]{\dfrac{\text{d}^{#3}#1}{\text{d}#2^{#3}}}
\newcommand{\pdn}[3]{\dfrac{\partial^{#3}#1}{\partial#2^{#3}}}
\newcommand{\p}[0]{^{\prime}}
\newcommand{\pp}[0]{^{\prime\prime}}
\newcommand{\op}[2][\text{L}]{#1 \left[ #2 \right]}

\newcommand{\lap}[1]{\mathcal{L}\left\{#1\right\}}
\newcommand{\lapinv}[1]{\mathcal{L}^{-1}\left\{#1\right\}}
\newcommand{\lapint}[1]{\int_0^\infty e^{-st}#1dt}
\newcommand{\evalat}[2]{\Big|_{#1}^{#2}}

\newcommand{\paren}[1]{ \left( #1 \right)}

\newcommand{\haxis}[4][\normcolor]{\draw[#1, <->] (-#2,0)--(#3,0) node[right]{$#4$}; }


\newcommand{\axis}[4]{\draw[\normcolor, <->] (-#1,0)--(#2,0) 
node[right]{$x$};
\draw[help lines, <->] (0,-#3)--(0,#4) node[above]{$y$};}

\newcommand{\laxis}[6]{\draw[<->] (-#1,0)--(#2,0) 
node[right]{$#5$};
\draw[ <->] (0,-#3)--(0,#4) node[above]{$#6$};}
\newcommand{\xcoord}[2]{
	\draw (#1,.2)--(#1,-.2) node[below]{$#2$};}
\newcommand{\textnode}[3]{
	\draw (#1,#2) node[below]{$#3$};}
	
\newcommand{\nxcoord}[2]{
	\draw (#1,-.2)--(#1,.2) node[above]{$#2$};}
\newcommand{\ycoord}[2]{
	\draw (.2,#1)--(-.2,#1) node[left]{$#2$};}
\newcommand{\nycoord}[2]{
	\draw (-.2,#1)--(.2,#1) node[right]{$#2$};}
\newcommand{\dlim}{\displaystyle\lim}
\newcommand{\dlimx}[1]{\displaystyle\lim_{x \rightarrow #1}}
\newcommand{\stickfig}[2]{
	\draw (#1,#2) arc(-90:270:2mm);
	\draw (#1,#2)--(#1,#2-.5) (#1-.25,#2-.75)--(#1,#2-.5)--(#1+.25,#2-.75) (#1-.2,#2-.2)--(#1+.2,#2-.2);}	

%\newcounter{example}
%\setcounter{example}{1}
%\newcounter{preFrameExample}
%\AtBeginEnvironment{frame}{\setcounter{preFrameExample}{\value{example}}}
%\newcommand{\ex}[1]{
%	 \setcounter{example}{\value{preFrameExample}}
%	 \textcolor{green}{\small\fbox{Example \arabic{example}: #1}}\\[8pt]
%	\stepcounter{example}}
%\newcommand{\exans}[1]{
%	\SUBTRACT{\value{preFrameExample}}{1}{\n}
%	 \textcolor{green}{\small\fbox{Solution \n: #1}}\\[8pt]}
\mode<presentation> {

% The Beamer class comes with a number of default slide themes
% which change the colors and layouts of slides. Below this is a list
% of all the themes, uncomment each in turn to see what they look like.


\usetheme{CambridgeUS}
\usecolortheme[named=black]{structure}


\newcommand{\studentcolor}[0]{ForestGreen}
\newcommand{\normcolor}[0]{NavyBlue}
\newcommand{\alertcolor}{Red}

\setbeamercolor{normal text}{fg=\normcolor}
\setbeamercolor{frametitle}{fg=\normcolor}
\setbeamercolor{section in head/foot}{fg=Black, bg=Gray!20}
\setbeamercolor{subsection in head/foot}{fg=Green!70!Black, bg=Gray!10}
\setbeamercolor{alerted text}{fg=\alertcolor}
\setbeamerfont{alerted text}{series=\bf}
\setbeamertemplate{enumerate items}[default]
\setbeamercolor{enumerate item}{fg=\normcolor}

\setbeamertemplate{footline} % To remove the footer line in all slides uncomment this line
%\setbeamertemplate{footline}[page number] % To replace the footer line in all slides with a simple slide count uncomment this line

\setbeamertemplate{navigation symbols}{} % To remove the navigation symbols from the bottom of all slides uncomment this line
}

\newcommand{\alertbox}[1]{\tcbox[on line, colframe=\alertcolor, colback=White, left=2pt,right=2pt,top=2pt,bottom=2pt]{\usebeamercolor*{normal text}#1}}


\newcommand{\startstu}{\setbeamercolor{normal text}{fg=\studentcolor}\usebeamercolor*{normal text}\setbeamercolor{enumerate item}{fg=\studentcolor}\usebeamercolor*{enumerate item}}
\newcommand{\stopstu}{\setbeamercolor{normal text}{fg=\normcolor}\usebeamercolor*{normal text}\setbeamercolor{enumerate item}{fg=\normcolor}\usebeamercolor*{enumerate item}}

\newcommand{\takenote}[1]{ \begin{collect}{notes}{}{}{}{}  #1  \end{collect}  \addtocounter{notestaken}{1}} %\ifthenelse{\value{notestaken}>0}{\hrulefill\\}{}

\makeatletter
\newcommand{\cover}{\alt{\beamer@makecovered}{\beamer@fakeinvisible}}
\newcommand{\ucover}[1]{\iftoggle{full}{}{\beamer@endcovered}\stopstu#1\startstu\iftoggle{full}{}{\beamer@startcovered}}
\makeatother

\newcommand{\skippause}{ \addtocounter{beamerpauses}{-1}}
\newcommand{\blockpres}{ \skippause \pause }

\newcommand{\studentify}[1]{\startstu #1  \stopstu }
\newcommand{\student}[1]{\iftoggle{full}{ \pause  \studentify{#1} }{\iftoggle{covered}{\studentify{#1}}{\cover{  #1 }}}}
\newcommand{\cstudent}[1]{\student{\begin{center} #1 \end{center}}}
\newcommand{\fullonly}[1]{\iftoggle{full}{ #1}{}}
\newcommand{\presentonly}[1]{\iftoggle{presentable}{ #1}{}}

\usepackage{xparse}
\usepackage{xifthen}

% shortcuts for commonly-used presentation elements
%\NewDocumentCommand{\slide}{o m}
% {\IfValueTF{#1}{\begin{frame}[t]{#1}}{\begin{frame}[t]} #2 \end{frame}}

\newtoggle{iscovered}

\newcommand{\slide}[2][]{%
%\setcounter{notestaken}{0}
\takenote{#2} 
%\ifthenelse{\equal{#1}{}}{\begin{frame}[t]}{\begin{frame}[t]{#1}} #2 \ifthenelse{\value{notestaken}>0}{ \note{\includecollection{notes}}}{} \end{frame}%
\ifthenelse{\equal{#1}{}}{\begin{frame}[t]}{\begin{frame}[t]{#1}} #2 \iftoggle{covered}{\settoggle{iscovered}{true}}{\settoggle{iscovered}{false}}  \note{ \iftoggle{iscovered}{}{\settoggle{covered}{true}} #2 \iftoggle{iscovered}{}{\settoggle{covered}{false}} } \end{frame}%
%\setcounter{notestaken}{0}
}
\newcommand{\defn}[2][]{%
 \setcounter{listcounter}{0}%
\ifthenelse{\equal{#1}{}}{\begin{block}{Definition}}{\begin{block}{#1 :}}%
 #2 \vspace{0.25em} \ifthenelse{\value{listcounter}>0}{\skippause}{} \pause \end{block}%
}



\newcommand{\arr}[2]{\begin{array}{#1}#2\end{array}}
\newcommand{\mat}[2]{\left[\arr{#1}{#2}\right]}
\newcommand{\carray}[1]{\arr{c}{#1}}
\newcommand{\larray}[1]{\arr{l}{#1}}
\newcommand{\rarray}[1]{\arr{r}{#1}}
\newcommand{\colvec}[1]{\mat{c}{#1}}

\newcommand{\itmz}[1]{\addtocounter{listcounter}{1} \begin{itemize}#1 \end{itemize} }
\newcommand{\subitem}[1]{\addtocounter{listcounter}{1} \begin{itemize} \item #1 \end{itemize}}
%
\newcommand{\enum}[1]{\addtocounter{listcounter}{1} \begin{enumerate} #1  \end{enumerate}  }


\newcommand{\algnlbl}[1]{\begin{align}#1  \end{align}} 
\newcommand{\algn}[1]{\begin{align*}#1  \end{align*}} 
\newcommand{\lgn}[1]{ \action<+->{#1} }
\newcommand{\slgn}[1]{\iftoggle{full}{\action<+->{ \startstu #1 \stopstu}}{ \cover{ #1 } } \takenote{$#1$}}

\newcommand{\chckmrk}{\alert{\checkmark}}

\usepackage{pifont}
\newcommand{\xmark}{\alert{\text{\large \ding{55}}}}

\newcommand{\return}[0]{\raisebox{.5ex}{\rotatebox[origin=c]{180}{$\Lsh$}}}
\usepackage{pbox}
%\newcommand{\ex}[1]{\rotatebox[origin=c]{10}{\uline{ex}}:$\;$\pbox[t][][b]{0.9\linewidth}{#1}}
\newcommand{\ex}[1]{\uline{ex}:$\;$\pbox[t][][t]{0.9\linewidth}{#1}}
\newcommand{\eg}[1]{e.g.,$\;$\pbox[t][][t]{0.9\linewidth}{#1}}
\newcommand{\tikzplot}[8][]{%
\begin{tikzpicture}

\begin{scope}[]%
\clip(-#2,-#4) rectangle (#3,#5);%
#8%
\end{scope}%
\laxis{#2}{#3}{#4}{#5}{#6}{#7}%
#1
\end{tikzpicture}%
}


\newcommand{\cancelslide}[1]{%
\begingroup%
\setbeamertemplate{background canvas}{%
\begin{tikzpicture}[remember picture,overlay]%
\draw[line width=2pt,red!60!black] %
  (current page.north west) -- (current page.south east);%
\draw[line width=2pt,red!60!black] %
  (current page.south west) -- (current page.north east);%
\end{tikzpicture}}%
#1%
\endgroup%
}
\renewcommand{\CancelColor}{\color{red}}
\newcommand{\twocols}[3][0.5]{\begin{columns}\begin{column}{#1\textwidth}#2\end{column}\hspace{1em}\vrule{}\hspace{1em}\begin{column}{#1\textwidth}#3\end{column}\end{columns}}

\newcommand{\twomini}[5][1]{\calculatespace \begin{minipage}[t]{\columnwidth}\begin{minipage}[][#1\contentheight][t]{#2\columnwidth}#4\end{minipage}\hfill\begin{minipage}[][#1\contentheight][t]{#3\columnwidth}#5\end{minipage}\end{minipage}}

\newcommand{\threemini}[7][1]{\calculatespace \begin{minipage}[t]{\columnwidth}\begin{minipage}[][#1\contentheight][t]{#2\columnwidth}#5\end{minipage}\hfill\begin{minipage}[][#1\contentheight][t]{#4\columnwidth}#6\end{minipage}\hfill\begin{minipage}[][#1\contentheight][t]{#3\columnwidth}#7\end{minipage}\end{minipage}}


\newcounter{listcounter}
\setcounter{listcounter}{0}



\newif\ifsidebartheme
\sidebarthemetrue

\newdimen\contentheight
\newdimen\contentwidth
\newdimen\contentleft
\newdimen\contentbottom
\makeatletter
\newcommand*{\calculatespace}{%
\contentheight=\paperheight%
\ifx\beamer@frametitle\@empty%
    \setbox\@tempboxa=\box\voidb@x%
  \else%
    \setbox\@tempboxa=\vbox{%
      \vbox{}%
      {\parskip0pt\usebeamertemplate***{frametitle}}%
    }%
    \ifsidebartheme%
      \advance\contentheight by-1em%
    \fi%
  \fi%
\advance\contentheight by-\ht\@tempboxa%
\advance\contentheight by-\dp\@tempboxa%
\advance\contentheight by-\beamer@frametopskip%
\ifbeamer@plainframe%
\contentbottom=0pt%
\else%
\advance\contentheight by-\headheight%
\advance\contentheight by\headdp%
\advance\contentheight by-\footheight%
\advance\contentheight by4pt%
\contentbottom=\footheight%
\advance\contentbottom by-4pt%
\fi%
\contentwidth=\paperwidth%
\ifbeamer@plainframe%
\contentleft=0pt%
\else%
\advance\contentwidth by-\beamer@rightsidebar%
\advance\contentwidth by-\beamer@leftsidebar\relax%
\contentleft=\beamer@leftsidebar%
\fi%
}
\makeatother



\iftoggle{dualscreen}{\setbeameroption{show notes on second screen=right}}{}
\usetikzlibrary{arrows}

\begin{document}
\settoggle{covered}{true}

\section{Lecture 15}
\subsection{Preamble}

\slide[Recall:]{
 \[\dd{}{t}\vec{x} = \mathbf{A} \vec{x} \qquad \text{with }\vec{x}(0) =\vec{x}_0 \]
Since $\mathbf{A}$ is an $n\times n$ matrix, our solutions live in $\mathbb{R}^n$
\vfill
Superposition:
\[\vec{x}(t) = c_1\vec{x}_1(t) +c_2 \vec{x}_2 + \dots + c_n \vec{x}_n(t)\]
\vfill
Each $\vec{x}_i(t)$ was found by solving an eigenproblem.\vfill
\student{\[\vec{x}_i (t)= e^{\lambda_i t} \vec{v}_i \qquad \text{where} \qquad A\vec{v}_i = \lambda_i \vec{v}_i\]}

}

\slide[Recall:]{

\[\vec{x}(t) = c_1\vec{x}_1(t) +c_2 \vec{x}_2 + \dots + c_n \vec{x}_n(t)\]
\vfill
The coefficients $c_i$ come from projecting the initial conditions $\vec{x}_0$ onto the basis
\[\left\{\vec{x}_1(0), \vec{x}_2(0), \dots, \vec{x}_n(0)  \right\}\]
\vfill
\student{
\[c_i = \vec{x}_0 \cdot \vec{x}_i(0) \qquad - \qquad \text{change of basis} \]\vfill
This an "inner product" defined on the vector space $\mathbb{R}^n$
\[\vec{a}\cdot \vec{b} = \left<\vec{a},\vec{b}\right> = \sum_i a_i \cdot b_i\]
}
}

\slide{For the rest of the semester, we will be using techniques based on inner products in a \uline{function space} - effectively an infinite dimensional vector space.
\vfill
\student{
Consider two functions $h(x)$ and $g(x)$ defined on an interval $[a,b]$, then the "standard" inner product is given by \[\left<h, g \right>=\intop_a^b h(x) g^*(x) dx\] where $g^*$ is the complex conjugate of $g$.}
}
\slide[]{
The two main techniques we will use are: \vfill
1) Laplace Transforms\vfill
\student{
\itmz{
\item ODE does not need to reduce to any specific eigenproblem.\vfill
\item Project ODE+ICs onto the space of exponential solutions. \subitem{ Use pattern recognition skills to convert projection back into a time-dependent solution.}\vfill
\item Very useful for ODEs with non-smooth forcing and understanding complicated modular systems.
}}\vfill
}
\settoggle{covered}{false}
\slide[]{
The two main techniques we will use are: \vfill
2) Fourier Series\vfill
\student{
\itmz{
\item ODE/PDE must reduce to specific eigenproblem(s)\vfill
\item Project ICs onto the space of periodic functions (sin/cos basis). \subitem{Projections can be directly interpreted as constant coefficients in an infinite series solution.}\vfill
\item We may also see some applications with hyperbolic trig. functions.
}}\vfill
}


\slide[\small Solving ODEs: $ay\pp +by\p+cy=h(t)$\hfill  with $\larray{y(0)=y_0\\y\p(0)=v_0}$]{
\vfill
\centerline{How do we "normally" solve this?}
\student{\[y=y_p+y_h\]}
\vfill
\enum{\item Solve homogeneous problem, get $y_h=c_1y_1+c_2y_2$. \vfill
\item Apply method of undetermined coefficients, get $y_p$ \vfill\student{\subitem{Requires knowing $h\p,h\pp,\dots$}}\vfill
\item Find $c_1$ \& $c_2$ by solving: \hspace{.5cm}$\carray{y_0=y_p(t_0)+c_1y_1(t_0)+c_2y_2(t_0) \\ v_0=y_p\p(t_0)+c_1y_1\p(t_0)+c_2y_2\p(t_0) }$\vfill
\item Finally \[y(t) =y_p+c_1y_1+c_2y_2\]
}
}

\newcommand{\interval}{\draw[|-|]}
\newcommand{\midnode}{node[midway, below]}
\slide[Non-smooth forcing]{\vspace{-1em}
Suppose some function $h(t)$ jumps abruptly between two values
\[\text{e.g., } h(t) = \begin{cases} 0 & t\leq a\\ 1 & t>a \end{cases}\]\vspace{-1em}

\tikzplot[\xcoord{3}{\text{kink}}; 

\interval (0,-.15) --(3,-.15) \midnode{$t\leq a$} ; 
\interval (3,-.15) --(5,-.15) \midnode{$t>a$} ; 


 ]{.1}{5}{.1}{1.5}{t}{\text{\small{y(t)}}}{
\draw (2,1) node{$y\pp+6y\p+y=h(t)$};
\draw[domain=0:2, thick, samples=100] plot ({\x*1.5}, {.8*exp(-\x)});
\draw[domain=2:10, thick, samples=100] plot ({\x*1.5}, {.8*exp(-\x)+(1-exp(-(\x-2)))*1.25});

}\hfil \tikzplot[\xcoord{3}{\text{jump}}

\interval (0,-.15) --(3,-.15) \midnode{$t\leq a$} ; 
\interval (3,-.15) --(5,-.15) \midnode{$t>a$} ; 
  ]{.1}{5}{.1}{1.5}{t}{\text{\small{y(t)}}}{
\draw (1.5,1.15) node{$y\p+6y=\dd{}{t}h(t)$};
\draw[domain=0:2, thick, samples=100] plot ({\x*1.5}, {.8*exp(-\x)});
\draw[domain=2:10, thick, samples=100] plot ({\x*1.5}, {.8*exp(-(\x-2))});
\draw[thick , dashed] (3,0.1)--(3,.8);
}
\student{Since the derivative of $h(t)$ is undefined at $t=a$, we cannot use the method of undetermined coefficients...\vfill
\centerline{We must use Laplace Transforms!}}
}

\subsection{Laplace Transforms}
\slide[Laplace Transforms]{
Suppose you have a real-valued function $y(t)$ defned for $t\in[0,\infty)$ that solves some ODE of interest.
\vfill
Its Laplace Transform is given by 


\algn{Y(s) = \lap{y(t)} &=  \left<e^{-st}, y(t) \right>\\\\&\student{=\int_0^\infty e^{-st} y(t)dt}}


}


\slide{\ex{$y(t)=\frac12$}\student{\hspace{3em}$\lap{y(t)}=Y(s)=\lapint{\frac12}$
\algn{&=-\frac{1}{2s}e^{-st}\Big|_0^\infty
&=-\lim_{A\to\infty}\frac{1}{2s}e^{-st}\Big|_0^A\\
&=-\frac{1}{2s}\lim_{A\to\infty}\paren{e^{-sA}-1}\\
&=\begin{cases}
\frac{1}{2s}& \text{if }s>0\\
DNE & \text{if }s\leq0
\end{cases}}
}\vfill
\centerline{
\tikzplot[\textnode{4.75}{1.5}{\overset{\mathcal{L}}{\rightarrow}}]{1}{3}{.1}{2.1}{\color{RubineRed}t}{\color{RubineRed}y(t)}{
\draw[domain=-1:6, smooth, RubineRed, thick, samples=150] plot ({\x}, {1.5});}
\hfill 
\tikzplot{1}{3}{.1}{2.1}{\color{YellowOrange}s}{\color{YellowOrange}Y(s)}{
\draw[domain=0.01:6, YellowOrange, thick, samples=100] plot ({\x}, {.02/(\x/5)});
\draw[pattern=north west lines, pattern color=\normcolor] (-2,-2) rectangle (0,4);}
}
}



\slide{
\ex{$y(t)=e^{-6t}$}\student{
\algn{\lap{y(t)}&=Y(s)=\lapint{e^{-6t}}\\
&\dots\\
&=\begin{cases}
\frac{1}{s+6} & \text{if }s>-6\\
DNE & \text{if }s\leq-6
\end{cases}
}}
\vfill
\centerline{
\tikzplot[\textnode{4.75}{1.5}{\overset{\mathcal{L}}{\rightarrow}}]{1}{3}{.1}{1.5}
{\color{RubineRed}t}
{\color{RubineRed}y(t)}{
\draw[domain=-1:6, smooth, RubineRed, thick, samples=150] plot ({\x}, {.4*exp(-\x)});}
\hfill 
\tikzplot[\xcoord{-.5}{-6}]{1}{3}{.1}{1.25}{\color{YellowOrange}s}{\color{YellowOrange}Y(s)}{
\draw[domain=-.499:6, YellowOrange, thick, samples=100] plot ({\x}, {.15/(\x+.5)});
\draw[pattern=north west lines, pattern color=\normcolor] (-2,-2) rectangle (-.5,4);}
}
}



\slide[Laplace Transform of Derivatives]{
Suppose $\lap{y(t)}=Y(s)$. What is $\lap{y\p(t)}?$\vfill \student{
\algn{\lap{y\p(t)}&=\int_0^\infty \underbrace{e^{-st}}_{u} \underbrace{y\p(t)dt}_{dv} &\arr{l}{v=y(t)\\du=-se^{-st}dt} \\
&=e^{-st}y(t)\evalat{0}{\infty}-(-s)\underbrace{\lapint{y(t)}}_{\lap{y(t)}}\\
&\overset{s>0}{=} - y(0)+s\lap{y(t)} \\\\&=  \boxed{sY(s)-y_0} } 
}\vfill
}%end slide

\slide[Laplace Transform of Derivatives]{
Given that $\lap{y\p(t)}=sY(s)-y_0$. What is $\lap{y\pp(t)}?$ \student{
\vfill
\algn{
\lap{y\pp(t)}&= s\lap{y\p(t)} - y\p(0)=s\left[ sY(s)-y_0 \right] -\underbrace{y\p(0)}_{v_0}\\\\
&=\boxed{s^2Y(s)-sy_0-v_0}}
}\vfill
}%end slide

\slide[Linearity of Laplace Transforms:]{
\vfill
\enum{\item $\lap{f(t)+g(t)}=\student{\lapint{\paren{f(t)+g(t)}}}$\\~\\
\student{$=\lap{f(t)}+\lap{g(t)}=F(s)+G(s)$}\vfill
\item $\lap{cf(t)}=\student{c\lap{f(t)}=cF(s)}$}

\vfill\ex{What is the Laplace transform of $y\p+6y$  with $y(0)=y_0$? }\vfill
\student{\algn{\lap{y\p+6y}&=\lap{y\p}+\lap{6y}\\
&=s\lap{y}-y_0+6\lap{y}\\
&=(s+6)Y(s)-y_0
}}
}
\slide{\ex{Find $Y(s)$ for $y\p+6y=3$  with $y(0)=y_0$.}\vfill
\student{
\algn{(s+6)Y(s)-y_0&=\lap{3}\\
\lap{y\p}+6\lap{y}&=\frac3s\\
sY(s)-y_0+6Y(s)&=\frac3s\\
Y(s)&=\underbrace{\frac{3}{s(s+6)}}_{\lap{???}}+\underbrace{\frac{y_0}{s+6}}_{y_0 \lap{e^{-6t}}}
}}\vfill}

\slide{\ex{Solve $y\p+6y=3$ with $y(0)=y_0$ using Laplace Transforms.}\vfill
\student{
Partial fraction decomposition
\algn{Y(s)=\frac{3}{s(s+6)}+\frac{y_0}{s+6}&=\frac{A}{s}+\frac{B}{s+6}+\frac{y_0}{s+6}\\
3+\cancel{y_0 \cdot s}&=A (s+6)+B\cdot s+\cancel{y_0 \cdot s}\\
3&=6A + (A+B)s\\
\intertext{True for all $s$ $\Rightarrow$ coefficients must match}
\text{\uline{constant terms}:}\quad 3&=6A &&A=\frac12\\
\text{\uline{$s$ terms}:}\quad 0&=A+B &&B=-A=-\frac12\\\\
Y(s) = \frac{1}{2s}-\frac{1}{2(s+6)} +\frac{y_0}{s+6}&=\frac{1}{2s}+\paren{y_0-\frac12}\frac{1}{s+6}
}

}\vfill}

\slide{

\student{
\algn{\ucover{Y(s)}&\ucover{=\frac{1}{2s}+\paren{y_0-\frac12}\frac{1}{s+6}}\\
&=F(s)+\paren{y_0-\frac12}G(s) & F(s)=\frac{1}{2s} \Rightarrow f(t)& = \frac12 \\
&& G(s)=\frac{1}{s+6} \Rightarrow g(t) &= e^{-6t}\\\
&=\lap{\frac12} + \paren{y_0-\frac12}\lap{e^{-6t}} \\
y(t)&=\frac{1}{2}+\paren{y_0-\frac12}e^{-6t}}
}\vfill
}

\slide[General Laplace Transform Method for IVPs]{\vspace{-1em}
\enum{\item Take Laplace transform of the entire DE \student{(ex. $y\p +6y=3$)}
\subitem{Use linearity and rules for transforming derivatives.}\vfill\student{\centerline{$sY(s)-y_0+6Y(s)=\frac3s$}}\vfill
\item Solve the resulting equation for $Y(s)$ \vfill
\student{\centerline{$Y(s)=\frac{3}{s(s+6)}+\frac{y_0}{s+6}$}}\vfill
\item Do some algebra to get a sum of  "easy" terms \vfill
\student{\centerline{$Y(s)=\frac{1}{2s}+\paren{y_0-\frac12}\frac{1}{s+6}$}}\vfill
\item Transform back from $Y(s)$ to $y(t)$ \subitem{Tackle each term in the sum individually.}\vfill
\student{\centerline{$y(t)=\frac{1}{2}+\paren{y_0-\frac12}e^{-6t}$}}\vfill  }
}

\slide[What are "easy" terms?]{
Spot a term in s-space $\rightarrow$ you now know what it corresponds to in t-space
\vfill
\algn{\lap{C}&=\frac{C}{s}&\text{Constant}\\
\lap{Ce^{-at}} & = \frac{C}{s+a}&\text{Exponential func.}\\
\lap{t^n}&=\frac{n!}{s^{n+1}}&\text{Power func.}\\
\lap{\sin \omega t}&=\frac{\omega}{\omega^2+s^2}\\
\lap{\cos \omega t}&=\frac{s}{\omega^2+s^2}\\
}\vfill
Tables exist with even more terms...

}
\slide{\ex{Use Laplace transforms to solve $y\pp+9y=0$ with $y(0)=1$ and $y\p(0)=-1$.}
\student{
\algn{s^2Y(s)-sy_0-v_0&+9Y(s)=0\\
(9+s^2)Y(s)&-s\cancelto{1}{y_0}-\cancelto{-1}{v_0} =0 \\
(9+s^2)Y(s)  &= -1 + s\\\\
Y(s) &=\frac{-1}{9+s^2} +\frac{s}{9+s^2}\\
&=A \frac{3}{9+s^2} +\frac{s}{9+s^2} && A\cdot3=-1\\
Y(s)&=\frac{-1}{3} \frac{3}{9+s^2} +\frac{s}{9+s^2} &&A=-\frac13\intertext{Invert the transform}
y(t)&=-\frac13 \sin(3t)+\cos(3t)
}
}
}


\slide[Summary]{

\itmz{
\item Laplace transform (LT): $\lap{f(t)}=\lapint{f(t)}$ \vfill
\item Maps $f(t) \rightarrow F(s)$ , from "t-space" to "s-space" \vfill
\item  LT is linear because the integral is linear \vfill
\item LT of derivatives (from integration by parts)
\algn{\lap{y\p(t)}&=sY(s)-y_0 \\
\lap{y\pp(t)}&=s^2Y(s)-sy_0-v_0} 
\item Using these we can take LT of constant coefficient DE’s \vfill
\item Solve the algebraic equation and invert the transform \vfill
\subitem{ Use tables of LT’s to do this}}
}
\end{document}

