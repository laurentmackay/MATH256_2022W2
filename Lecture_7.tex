\documentclass[11pt, dvipsnames, handout]{beamer}
\newtoggle{full}
\settoggle{full}{true}

\newtoggle{covered}
\settoggle{covered}{false}

\newtoggle{presentable}
\settoggle{presentable}{false}

\newtoggle{dualscreen}
\settoggle{dualscreen}{false}

\usepackage{pgfplots}
%\pgfplotsset{compat = newest}

\usepackage{pgfpages}

\setbeamertemplate{note page}{\pagecolor{yellow!5}\vfill \insertnote \vfill}
\usepackage{collect}
\definecollection{notes}
\newcounter{notestaken}

\usepackage{xpatch}

\usepackage{ulem}

\usepackage[framemethod=tikz]{mdframed}

\usepackage{scalerel}
\usepackage{calc}

%\usepackage{enumitem}
\setlength\fboxsep{.2em}

\usepackage{graphicx} % Allows including images
\usepackage{booktabs} % Allows the use of \toprule, \midrule and \bottomrule in tables

\xpatchcmd{\itemize}
  {\def\makelabel}
  {\setlength{\itemsep}{0.65 em}\def\makelabel}
  {}
  {}


\xpatchcmd{\beamer@enum@}
  {\def\makelabel}
  {\setlength{\itemsep}{0.65 em}\def\makelabel}
  {}
  {}


%\makeatletter
%\renewcommand{\itemize}[1][]{%
%  \beamer@ifempty{#1}{}{\def\beamer@defaultospec{#1}}%
%  \ifnum \@itemdepth >2\relax\@toodeep\else
%    \advance\@itemdepth\@ne
%    \beamer@computepref\@itemdepth% sets \beameritemnestingprefix
%    \usebeamerfont{itemize/enumerate \beameritemnestingprefix body}%
%    \usebeamercolor[fg]{itemize/enumerate \beameritemnestingprefix body}%
%    \usebeamertemplate{itemize/enumerate \beameritemnestingprefix body begin}%
%    \list
%      {\usebeamertemplate{itemize \beameritemnestingprefix item}}
%      {%
%        \setlength\topsep{1em}%NEW
%        \setlength\partopsep{1em}%NEW
%        \setlength\itemsep{1em}%NEW
%        \def\makelabel##1{%
%          {%
%            \hss\llap{{%
%                \usebeamerfont*{itemize \beameritemnestingprefix item}%
%                \usebeamercolor[fg]{itemize \beameritemnestingprefix item}##1}}%
%          }%
%        }%
%      }
%  \fi%
%  \beamer@cramped%
%  \raggedright%
%  \beamer@firstlineitemizeunskip%
%}
%
%
%
%
%
%\makeatother

%\setlist[beamer@enum@]{topsep=1 em}
%\let\origcheckmark\checkmark %screw you dingbat
%\let\checkmark\undefined %screw you dingbat
%\usepackage{dingbat} 
%\let\checkmark\origcheckmark %screw you dingbat






%\usepackage{fontawesome}

\usepackage{mathtools}
\usepackage{etoolbox, calculator}

\usepackage{xcolor}
\usepackage{tikz}
\usetikzlibrary{arrows.meta}
\usetikzlibrary{calc}
\usepackage[nomessages]{fp}
\usepackage{transparent}
\usepackage{accsupp}
%\usepackage{color, xcolor}

%colorblind-friendly palette
%\definecolor{dblue}{RGB}{51,34,136}
\definecolor{lblue}{RGB}{136,204,238}
%\definecolor{green}{RGB}{17,119,51}
\definecolor{tan}{RGB}{221,204,119}
%\definecolor{mauve}{RGB}{204,102,119}

\usepackage{tcolorbox}



\usepackage{xifthen}
\usepackage{nicefrac}
\usepackage{amsmath}
\usepackage{amsthm}
\usepackage{amssymb}
\theoremstyle{definition}
\newtheorem*{define}{Definition}
\newtheorem*{recall}{Recall}


\DeclareMathOperator{\tr}{tr}

\usepackage{multicol}
%\setlength{\columnsep}{1cm}

\usepackage{tablists, amsmath,vwcol, cancel, polynom}
\usetikzlibrary{shapes, patterns, decorations.shapes}
%\usepackage{tikzpeople}
\tikzstyle{vertex}=[shape=circle, minimum size=2mm, inner sep=0, fill]
\tikzstyle{opendot}=[shape=circle, minimum size=2mm, inner sep=0, fill=white, draw]

% common math quick commands
\newcommand{\nicedd}[2]{\nicefrac{\text{d}#1}{\text{d}#2}}
\newcommand{\dd}[2]{\dfrac{\text{d}#1}{\text{d}#2}}
\newcommand{\pd}[2]{\dfrac{\partial #1}{\partial#2}}
\renewcommand{\d}[1]{\text{d}#1}
\newcommand{\ddn}[3]{\dfrac{\text{d}^{#3}#1}{\text{d}#2^{#3}}}
\newcommand{\pdn}[3]{\dfrac{\partial^{#3}#1}{\partial#2^{#3}}}
\newcommand{\p}[0]{^{\prime}}
\newcommand{\pp}[0]{^{\prime\prime}}
\newcommand{\op}[2][\text{L}]{#1 \left[ #2 \right]}

\newcommand{\lap}[1]{\mathcal{L}\left\{#1\right\}}
\newcommand{\lapinv}[1]{\mathcal{L}^{-1}\left\{#1\right\}}
\newcommand{\lapint}[1]{\int_0^\infty e^{-st}#1dt}
\newcommand{\evalat}[2]{\Big|_{#1}^{#2}}

\newcommand{\paren}[1]{ \left( #1 \right)}

\newcommand{\haxis}[4][\normcolor]{\draw[#1, <->] (-#2,0)--(#3,0) node[right]{$#4$}; }


\newcommand{\axis}[4]{\draw[\normcolor, <->] (-#1,0)--(#2,0) 
node[right]{$x$};
\draw[help lines, <->] (0,-#3)--(0,#4) node[above]{$y$};}

\newcommand{\laxis}[6]{\draw[<->] (-#1,0)--(#2,0) 
node[right]{$#5$};
\draw[ <->] (0,-#3)--(0,#4) node[above]{$#6$};}
\newcommand{\xcoord}[2]{
	\draw (#1,.2)--(#1,-.2) node[below]{$#2$};}
\newcommand{\textnode}[3]{
	\draw (#1,#2) node[below]{$#3$};}
	
\newcommand{\nxcoord}[2]{
	\draw (#1,-.2)--(#1,.2) node[above]{$#2$};}
\newcommand{\ycoord}[2]{
	\draw (.2,#1)--(-.2,#1) node[left]{$#2$};}
\newcommand{\nycoord}[2]{
	\draw (-.2,#1)--(.2,#1) node[right]{$#2$};}
\newcommand{\dlim}{\displaystyle\lim}
\newcommand{\dlimx}[1]{\displaystyle\lim_{x \rightarrow #1}}
\newcommand{\stickfig}[2]{
	\draw (#1,#2) arc(-90:270:2mm);
	\draw (#1,#2)--(#1,#2-.5) (#1-.25,#2-.75)--(#1,#2-.5)--(#1+.25,#2-.75) (#1-.2,#2-.2)--(#1+.2,#2-.2);}	

%\newcounter{example}
%\setcounter{example}{1}
%\newcounter{preFrameExample}
%\AtBeginEnvironment{frame}{\setcounter{preFrameExample}{\value{example}}}
%\newcommand{\ex}[1]{
%	 \setcounter{example}{\value{preFrameExample}}
%	 \textcolor{green}{\small\fbox{Example \arabic{example}: #1}}\\[8pt]
%	\stepcounter{example}}
%\newcommand{\exans}[1]{
%	\SUBTRACT{\value{preFrameExample}}{1}{\n}
%	 \textcolor{green}{\small\fbox{Solution \n: #1}}\\[8pt]}
\mode<presentation> {

% The Beamer class comes with a number of default slide themes
% which change the colors and layouts of slides. Below this is a list
% of all the themes, uncomment each in turn to see what they look like.


\usetheme{CambridgeUS}
\usecolortheme[named=black]{structure}


\newcommand{\studentcolor}[0]{ForestGreen}
\newcommand{\normcolor}[0]{NavyBlue}
\newcommand{\alertcolor}{Red}

\setbeamercolor{normal text}{fg=\normcolor}
\setbeamercolor{frametitle}{fg=\normcolor}
\setbeamercolor{section in head/foot}{fg=Black, bg=Gray!20}
\setbeamercolor{subsection in head/foot}{fg=Green!70!Black, bg=Gray!10}
\setbeamercolor{alerted text}{fg=\alertcolor}
\setbeamerfont{alerted text}{series=\bf}
\setbeamertemplate{enumerate items}[default]
\setbeamercolor{enumerate item}{fg=\normcolor}

\setbeamertemplate{footline} % To remove the footer line in all slides uncomment this line
%\setbeamertemplate{footline}[page number] % To replace the footer line in all slides with a simple slide count uncomment this line

\setbeamertemplate{navigation symbols}{} % To remove the navigation symbols from the bottom of all slides uncomment this line
}

\newcommand{\alertbox}[1]{\tcbox[on line, colframe=\alertcolor, colback=White, left=2pt,right=2pt,top=2pt,bottom=2pt]{\usebeamercolor*{normal text}#1}}


\newcommand{\startstu}{\setbeamercolor{normal text}{fg=\studentcolor}\usebeamercolor*{normal text}\setbeamercolor{enumerate item}{fg=\studentcolor}\usebeamercolor*{enumerate item}}
\newcommand{\stopstu}{\setbeamercolor{normal text}{fg=\normcolor}\usebeamercolor*{normal text}\setbeamercolor{enumerate item}{fg=\normcolor}\usebeamercolor*{enumerate item}}

\newcommand{\takenote}[1]{ \begin{collect}{notes}{}{}{}{}  #1  \end{collect}  \addtocounter{notestaken}{1}} %\ifthenelse{\value{notestaken}>0}{\hrulefill\\}{}

\makeatletter
\newcommand{\cover}{\alt{\beamer@makecovered}{\beamer@fakeinvisible}}
\newcommand{\ucover}[1]{\iftoggle{full}{}{\beamer@endcovered}\stopstu#1\startstu\iftoggle{full}{}{\beamer@startcovered}}
\makeatother

\newcommand{\skippause}{ \addtocounter{beamerpauses}{-1}}
\newcommand{\blockpres}{ \skippause \pause }

\newcommand{\studentify}[1]{\startstu #1  \stopstu }
\newcommand{\student}[1]{\iftoggle{full}{ \pause  \studentify{#1} }{\iftoggle{covered}{\studentify{#1}}{\cover{  #1 }}}}
\newcommand{\cstudent}[1]{\student{\begin{center} #1 \end{center}}}
\newcommand{\fullonly}[1]{\iftoggle{full}{ #1}{}}
\newcommand{\presentonly}[1]{\iftoggle{presentable}{ #1}{}}

\usepackage{xparse}
\usepackage{xifthen}

% shortcuts for commonly-used presentation elements
%\NewDocumentCommand{\slide}{o m}
% {\IfValueTF{#1}{\begin{frame}[t]{#1}}{\begin{frame}[t]} #2 \end{frame}}

\newtoggle{iscovered}

\newcommand{\slide}[2][]{%
%\setcounter{notestaken}{0}
\takenote{#2} 
%\ifthenelse{\equal{#1}{}}{\begin{frame}[t]}{\begin{frame}[t]{#1}} #2 \ifthenelse{\value{notestaken}>0}{ \note{\includecollection{notes}}}{} \end{frame}%
\ifthenelse{\equal{#1}{}}{\begin{frame}[t]}{\begin{frame}[t]{#1}} #2 \iftoggle{covered}{\settoggle{iscovered}{true}}{\settoggle{iscovered}{false}}  \note{ \iftoggle{iscovered}{}{\settoggle{covered}{true}} #2 \iftoggle{iscovered}{}{\settoggle{covered}{false}} } \end{frame}%
%\setcounter{notestaken}{0}
}
\newcommand{\defn}[2][]{%
 \setcounter{listcounter}{0}%
\ifthenelse{\equal{#1}{}}{\begin{block}{Definition}}{\begin{block}{#1 :}}%
 #2 \vspace{0.25em} \ifthenelse{\value{listcounter}>0}{\skippause}{} \pause \end{block}%
}



\newcommand{\arr}[2]{\begin{array}{#1}#2\end{array}}
\newcommand{\mat}[2]{\left[\arr{#1}{#2}\right]}
\newcommand{\carray}[1]{\arr{c}{#1}}
\newcommand{\larray}[1]{\arr{l}{#1}}
\newcommand{\rarray}[1]{\arr{r}{#1}}
\newcommand{\colvec}[1]{\mat{c}{#1}}

\newcommand{\itmz}[1]{\addtocounter{listcounter}{1} \begin{itemize}#1 \end{itemize} }
\newcommand{\subitem}[1]{\addtocounter{listcounter}{1} \begin{itemize} \item #1 \end{itemize}}
%
\newcommand{\enum}[1]{\addtocounter{listcounter}{1} \begin{enumerate} #1  \end{enumerate}  }


\newcommand{\algnlbl}[1]{\begin{align}#1  \end{align}} 
\newcommand{\algn}[1]{\begin{align*}#1  \end{align*}} 
\newcommand{\lgn}[1]{ \action<+->{#1} }
\newcommand{\slgn}[1]{\iftoggle{full}{\action<+->{ \startstu #1 \stopstu}}{ \cover{ #1 } } \takenote{$#1$}}

\newcommand{\chckmrk}{\alert{\checkmark}}

\usepackage{pifont}
\newcommand{\xmark}{\alert{\text{\large \ding{55}}}}

\newcommand{\return}[0]{\raisebox{.5ex}{\rotatebox[origin=c]{180}{$\Lsh$}}}
\usepackage{pbox}
%\newcommand{\ex}[1]{\rotatebox[origin=c]{10}{\uline{ex}}:$\;$\pbox[t][][b]{0.9\linewidth}{#1}}
\newcommand{\ex}[1]{\uline{ex}:$\;$\pbox[t][][t]{0.9\linewidth}{#1}}
\newcommand{\eg}[1]{e.g.,$\;$\pbox[t][][t]{0.9\linewidth}{#1}}
\newcommand{\tikzplot}[8][]{%
\begin{tikzpicture}

\begin{scope}[]%
\clip(-#2,-#4) rectangle (#3,#5);%
#8%
\end{scope}%
\laxis{#2}{#3}{#4}{#5}{#6}{#7}%
#1
\end{tikzpicture}%
}


\newcommand{\cancelslide}[1]{%
\begingroup%
\setbeamertemplate{background canvas}{%
\begin{tikzpicture}[remember picture,overlay]%
\draw[line width=2pt,red!60!black] %
  (current page.north west) -- (current page.south east);%
\draw[line width=2pt,red!60!black] %
  (current page.south west) -- (current page.north east);%
\end{tikzpicture}}%
#1%
\endgroup%
}
\renewcommand{\CancelColor}{\color{red}}
\newcommand{\twocols}[3][0.5]{\begin{columns}\begin{column}{#1\textwidth}#2\end{column}\hspace{1em}\vrule{}\hspace{1em}\begin{column}{#1\textwidth}#3\end{column}\end{columns}}

\newcommand{\twomini}[5][1]{\calculatespace \begin{minipage}[t]{\columnwidth}\begin{minipage}[][#1\contentheight][t]{#2\columnwidth}#4\end{minipage}\hfill\begin{minipage}[][#1\contentheight][t]{#3\columnwidth}#5\end{minipage}\end{minipage}}

\newcommand{\threemini}[7][1]{\calculatespace \begin{minipage}[t]{\columnwidth}\begin{minipage}[][#1\contentheight][t]{#2\columnwidth}#5\end{minipage}\hfill\begin{minipage}[][#1\contentheight][t]{#4\columnwidth}#6\end{minipage}\hfill\begin{minipage}[][#1\contentheight][t]{#3\columnwidth}#7\end{minipage}\end{minipage}}


\newcounter{listcounter}
\setcounter{listcounter}{0}



\newif\ifsidebartheme
\sidebarthemetrue

\newdimen\contentheight
\newdimen\contentwidth
\newdimen\contentleft
\newdimen\contentbottom
\makeatletter
\newcommand*{\calculatespace}{%
\contentheight=\paperheight%
\ifx\beamer@frametitle\@empty%
    \setbox\@tempboxa=\box\voidb@x%
  \else%
    \setbox\@tempboxa=\vbox{%
      \vbox{}%
      {\parskip0pt\usebeamertemplate***{frametitle}}%
    }%
    \ifsidebartheme%
      \advance\contentheight by-1em%
    \fi%
  \fi%
\advance\contentheight by-\ht\@tempboxa%
\advance\contentheight by-\dp\@tempboxa%
\advance\contentheight by-\beamer@frametopskip%
\ifbeamer@plainframe%
\contentbottom=0pt%
\else%
\advance\contentheight by-\headheight%
\advance\contentheight by\headdp%
\advance\contentheight by-\footheight%
\advance\contentheight by4pt%
\contentbottom=\footheight%
\advance\contentbottom by-4pt%
\fi%
\contentwidth=\paperwidth%
\ifbeamer@plainframe%
\contentleft=0pt%
\else%
\advance\contentwidth by-\beamer@rightsidebar%
\advance\contentwidth by-\beamer@leftsidebar\relax%
\contentleft=\beamer@leftsidebar%
\fi%
}
\makeatother



\iftoggle{dualscreen}{\setbeameroption{show notes on second screen=right}}{}


\begin{document}

\section{Lecture 7}
\subsection{Introduction}


\slide[Recall: Constant Coefficients 2$^{nd}$ Order Homogeneous IVP]{
\[ay\pp +by\p +cy= h(t) =0 \qquad \carray {y(0)=y_0\\ y\p (0)=v_0}\]
\ex{Spring-dashpot with no external forcing}
\[y_h(t)=c_1y_1(t)+c_2y_2(t) \qquad -\qquad  \text{three major cases}\]
What about for the inhomogeneous case ($h(t)\neq0$) ?
\vfill
General solution:
\[y_g = y_p +y_h\]
\vfill\student{
Recall, $y_p$ has no arbitrary coefficients.
\vfill
How can we find $y_p$?}

}

\slide[$ay\pp +by\p +cy = h(t) \neq 0 $]{
Two major cases:
\student{
\enum{
\item $c=0$\vfill
Define $v(t)=y\p$
\[av\p+bv=h(t)\]
Solve by method of integrating factors
\item $c\neq0$\vfill
\centerline{Method of Undetermined Coefficients}
}
}
}

\slide[$ay\pp +by\p +cy = h(t) \neq 0 $]{
$c\neq0$ \hfill -\hfill Method of Undetermined Coefficients\vfill
\student{
Basic idea
\[y_g = y_p +y_h\]
We know how to find $y_h$\subitem{ it makes the LHS=0}\vfill

When we plug $y_p$ into the LHS, it must equal $h(t)$.
\subitem{$y_p$ + its derivatives = $h(t)$}\vfill

Idea: differentiate $h(t)$ a bunch and see what kind of funtion $y_p$ could be.

}
}

\slide[Find all the functional forms obtained by differentiation]{
$h(t)=t^2$
\student{\[\{  t^2, t, \text{constant}\} \qquad \text{finite set}\]}
\vfill
$h(t)=\cos(3t)$
\student{\[\left\{ \sin(3t), \cos(3t)\right\} \qquad \text{finite set} \]}
\vfill
$h(t)=\ln(t)$
\student{\[\left\{ \ln(t), \frac1t, \frac{1}{t^2}, \frac{1}{t^3}, \dots \right\} \qquad \text{infinite set}\]}
}

\slide[Find the particular solution to $3y\pp+2y\p+y=t^2$]{\vspace{-0.75em}
Hint: guess $y_p=A+Bt+Ct^2$
\student{\algn{
y_p\p&=B+2Ct & y_p\pp&=2C\intertext{plug into ODE}
6C+&2(B+2Ct)+A+Bt+Ct^2=t^2\\
&Ct^2 + (4C+B)t+(A+2B+6C)=t^2
}
We need this to be true for all $t$. Equate the coefficients in front of the different time-dependent functions.
\algn{\uline{t^2}: C &= 1 & \uline{t}:  (4C+B)&=0\\
\uline{\text{constant}}: A+2B+6C&=0 &4+B&=0& B=-4\\
A-8+6&=0&\Rightarrow A=2}\vfill
\[y_p=2-4t+t^2\]
}

}

\slide[Find the particular solution to $y\pp+3y=24e^{3t}$]{\vspace{-0.75em}
\student{
Guess $y_p(t)=Ae^{3t}$
\algn{
y_p\p&=3Ae^{3t}& y_p\pp&=9Ae^{3t}\intertext{plug into ODE}
9Ae^{3t}+3Ae^{3t}&=24e^{3t}\\
12Ae^{3t}&=24e^{3t}
}
We need this to be true for all $t$. Equate the coefficients in front of the different time-dependent functions.
\algn{\uline{e^{3t}}: 12A &= 24 & A&=2\\\\
y_p(t)&=2e^{3t}
}
}
}

\slide[$ay\pp +by\p +cy = h(t) \neq 0 $]{
Will that approach always work?\vfill
\student{
Yes, if the function $h(t)$ has a finite family of derivative functional forms.
\vfill
\ex{$e^{t}, t^8, te^t, \cos(t), \sin(t), \dots$}
\vfill
No, if it has infinitely many derivatives...\vfill
\ex{$\ln(t), t^{-1}, t^{-2},\dots$}

}}

\subsection{Method of Undetermined Coefficients}
\slide[\begin{minipage}{.5\textwidth}Method of Undetermined\\ Coefficients:\end{minipage}\hfill\begin{minipage}{.34\textwidth}$\rarray{ay\pp+by\p+cy=h(t)\\y(t)=y_p(t)+y_h(t)}$\end{minipage}]{

\small
\vfill
\begin{table}\setlength\extrarowheight{5pt}
\begin{tabular}{|c|c|}
\hline 
Form of function $h(t)$ & Geuss for $y_{p}\left(t\right)$\tabularnewline
\hline 
\hline 
$\sum_{j=0}^{N}B_{j}t^{j}$ & $\sum_{j=0}^{N}A_{j}t^{j}$\tabularnewline
\hline 
$e^{\lambda t}$ & $Ae^{\lambda t}$\tabularnewline
\hline 
$\sin \omega t$ or $\cos  \omega t$ & $A\sin \omega t+B\cos \omega t$\tabularnewline
\hline 
$e^{\lambda t}\sin \omega t$ or $e^{\lambda t}\cos \omega t$ & $e^{\lambda t}A\sin \omega t+e^{ \lambda t}B\cos \omega t$\tabularnewline
\hline 
Additive combinations of above & Additive combinations of above\tabularnewline
\hline 
Multiplicative combinations of above & Multiplicative combinations of above\tabularnewline
\hline 
Part of the homogeneous solution {\tiny Note}\footnotemark& $Ath(t) \text{ or } At^2h(t)$ \text{ or } \dots \tabularnewline
\hline 
Anything else & You are out of luck\tabularnewline
\hline 
\end{tabular}
\end{table}\vfill

\footnotetext[1]{Note: This corresponds to resonance. }
\footnotetext[2]{Note: $b_j$ , $c_j$ , $b$, $c$, $A$, and $B$ are all constants in the above table}


}

\slide{Find the particular solution to $y\pp -6y\p+\frac{25}{4}y =3te^{2t}$\vfill
\student{
guess $y_p = Ae^{2t}+Bte^{2t}$
\algn{y_p\p&=2Ae^{2t}+Be^{2t} +2Bte^{2t} &y_p\pp&=2(2A+B)e^{2t} + 2Be^{2t}+4Bte^{2t}\\
&=(2A+B)e^{2t}+2Bte^{2t}&&=4(A+B)e^{2t} + 4 B t e^{2t}\intertext{plug into DE}
}\vspace{-3em}
\algn{4(A+B)e^{2t} + 4 B t e^{2t} - 6(2A+B)e^{2t}-12Bte^{2t} + \frac{25}{4}Ae^{2t}+\frac{25}{4}&Bte^{2t}&\\&=3te^{2t}}
\vspace{-2em}
\algn{
-\frac74Bte^{2t} -& \frac14\paren{7A+8B}e^{2t}=3te^{2t} \\
\uline{te^{2t}}:  -\frac74B&=3  & B&=-\frac{12}{7}\\
\uline{e^{2t}}:  7 A + 8 B &=0  & A&=\frac{8\cdot12}{7\cdot7}=\frac{96}{49}}\vfill
\[y_p =\frac{96}{49}e^{2t} -\frac{12}{7}te^{2t} \]
}
}

\slide{Solve the IVP $y\pp -3y\p -4y =3e^{2t}$ with $ y(0)=y\p(0)=1$
\student{
\[ y_g=y_p+c_1y_1+c_2y_2 \]
\algn{
y_{1,2}=e^{rt}\quad & \Rightarrow \quad (r^2-3r-4)=0\\
r_{1,2} &= \frac{3\pm\sqrt{9+16}}{2} =\frac{3\pm 5}{2} = 4,-1 \\\\
 y_g &= c_1 e^{4t} + c_2 e^{-t} -\frac12 e^{2t}\\
 y_g\p &= 4 c_1 e^{4t} - c_2 e^{-t} - e^{2t}\intertext{Initial Conditions:}
y(0)=1=c_1 + c_2 -\frac12\\
y\p(0)=1 = 4c_1-c_2-1
}

}
}
\slide{\vspace{-2em}\algn{y(0)=1=c_1 + c_2 -\frac12\\
y\p(0)=1 = 4c_1-c_2-1}
\student{\vspace{-2em}
\algn{\intertext{Add the two equations}
2&=5c_1-\frac32& 
\frac{7}{2} &= 5c_1 \\ c_1 &=  \frac{7}{10} \\
1&=\frac{7}{10} + c_2 -\frac{5}{10} &
1&=\frac{2}{10}+c_2\\
c_2&=\frac{8}{10}\\\\
y(t)&=\frac{7}{10}e^{4t} + \frac{8}{10}e^{-t} -\frac12e^{2t}
}
}
}
 



\end{document}