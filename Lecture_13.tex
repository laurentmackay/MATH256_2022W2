\documentclass[11pt, dvipsnames, handout]{beamer}
\newtoggle{full}
\settoggle{full}{true}

\newtoggle{covered}
\settoggle{covered}{false}

\newtoggle{presentable}
\settoggle{presentable}{false}

\newtoggle{dualscreen}
\settoggle{dualscreen}{false}

\usepackage{pgfplots}
%\pgfplotsset{compat = newest}

\usepackage{pgfpages}

\setbeamertemplate{note page}{\pagecolor{yellow!5}\vfill \insertnote \vfill}
\usepackage{collect}
\definecollection{notes}
\newcounter{notestaken}

\usepackage{xpatch}

\usepackage{ulem}

\usepackage[framemethod=tikz]{mdframed}

\usepackage{scalerel}
\usepackage{calc}

%\usepackage{enumitem}
\setlength\fboxsep{.2em}

\usepackage{graphicx} % Allows including images
\usepackage{booktabs} % Allows the use of \toprule, \midrule and \bottomrule in tables

\xpatchcmd{\itemize}
  {\def\makelabel}
  {\setlength{\itemsep}{0.65 em}\def\makelabel}
  {}
  {}


\xpatchcmd{\beamer@enum@}
  {\def\makelabel}
  {\setlength{\itemsep}{0.65 em}\def\makelabel}
  {}
  {}


%\makeatletter
%\renewcommand{\itemize}[1][]{%
%  \beamer@ifempty{#1}{}{\def\beamer@defaultospec{#1}}%
%  \ifnum \@itemdepth >2\relax\@toodeep\else
%    \advance\@itemdepth\@ne
%    \beamer@computepref\@itemdepth% sets \beameritemnestingprefix
%    \usebeamerfont{itemize/enumerate \beameritemnestingprefix body}%
%    \usebeamercolor[fg]{itemize/enumerate \beameritemnestingprefix body}%
%    \usebeamertemplate{itemize/enumerate \beameritemnestingprefix body begin}%
%    \list
%      {\usebeamertemplate{itemize \beameritemnestingprefix item}}
%      {%
%        \setlength\topsep{1em}%NEW
%        \setlength\partopsep{1em}%NEW
%        \setlength\itemsep{1em}%NEW
%        \def\makelabel##1{%
%          {%
%            \hss\llap{{%
%                \usebeamerfont*{itemize \beameritemnestingprefix item}%
%                \usebeamercolor[fg]{itemize \beameritemnestingprefix item}##1}}%
%          }%
%        }%
%      }
%  \fi%
%  \beamer@cramped%
%  \raggedright%
%  \beamer@firstlineitemizeunskip%
%}
%
%
%
%
%
%\makeatother

%\setlist[beamer@enum@]{topsep=1 em}
%\let\origcheckmark\checkmark %screw you dingbat
%\let\checkmark\undefined %screw you dingbat
%\usepackage{dingbat} 
%\let\checkmark\origcheckmark %screw you dingbat






%\usepackage{fontawesome}

\usepackage{mathtools}
\usepackage{etoolbox, calculator}

\usepackage{xcolor}
\usepackage{tikz}
\usetikzlibrary{arrows.meta}
\usetikzlibrary{calc}
\usepackage[nomessages]{fp}
\usepackage{transparent}
\usepackage{accsupp}
%\usepackage{color, xcolor}

%colorblind-friendly palette
%\definecolor{dblue}{RGB}{51,34,136}
\definecolor{lblue}{RGB}{136,204,238}
%\definecolor{green}{RGB}{17,119,51}
\definecolor{tan}{RGB}{221,204,119}
%\definecolor{mauve}{RGB}{204,102,119}

\usepackage{tcolorbox}



\usepackage{xifthen}
\usepackage{nicefrac}
\usepackage{amsmath}
\usepackage{amsthm}
\usepackage{amssymb}
\theoremstyle{definition}
\newtheorem*{define}{Definition}
\newtheorem*{recall}{Recall}


\DeclareMathOperator{\tr}{tr}

\usepackage{multicol}
%\setlength{\columnsep}{1cm}

\usepackage{tablists, amsmath,vwcol, cancel, polynom}
\usetikzlibrary{shapes, patterns, decorations.shapes}
%\usepackage{tikzpeople}
\tikzstyle{vertex}=[shape=circle, minimum size=2mm, inner sep=0, fill]
\tikzstyle{opendot}=[shape=circle, minimum size=2mm, inner sep=0, fill=white, draw]

% common math quick commands
\newcommand{\nicedd}[2]{\nicefrac{\text{d}#1}{\text{d}#2}}
\newcommand{\dd}[2]{\dfrac{\text{d}#1}{\text{d}#2}}
\newcommand{\pd}[2]{\dfrac{\partial #1}{\partial#2}}
\renewcommand{\d}[1]{\text{d}#1}
\newcommand{\ddn}[3]{\dfrac{\text{d}^{#3}#1}{\text{d}#2^{#3}}}
\newcommand{\pdn}[3]{\dfrac{\partial^{#3}#1}{\partial#2^{#3}}}
\newcommand{\p}[0]{^{\prime}}
\newcommand{\pp}[0]{^{\prime\prime}}
\newcommand{\op}[2][\text{L}]{#1 \left[ #2 \right]}

\newcommand{\lap}[1]{\mathcal{L}\left\{#1\right\}}
\newcommand{\lapinv}[1]{\mathcal{L}^{-1}\left\{#1\right\}}
\newcommand{\lapint}[1]{\int_0^\infty e^{-st}#1dt}
\newcommand{\evalat}[2]{\Big|_{#1}^{#2}}

\newcommand{\paren}[1]{ \left( #1 \right)}

\newcommand{\haxis}[4][\normcolor]{\draw[#1, <->] (-#2,0)--(#3,0) node[right]{$#4$}; }


\newcommand{\axis}[4]{\draw[\normcolor, <->] (-#1,0)--(#2,0) 
node[right]{$x$};
\draw[help lines, <->] (0,-#3)--(0,#4) node[above]{$y$};}

\newcommand{\laxis}[6]{\draw[<->] (-#1,0)--(#2,0) 
node[right]{$#5$};
\draw[ <->] (0,-#3)--(0,#4) node[above]{$#6$};}
\newcommand{\xcoord}[2]{
	\draw (#1,.2)--(#1,-.2) node[below]{$#2$};}
\newcommand{\textnode}[3]{
	\draw (#1,#2) node[below]{$#3$};}
	
\newcommand{\nxcoord}[2]{
	\draw (#1,-.2)--(#1,.2) node[above]{$#2$};}
\newcommand{\ycoord}[2]{
	\draw (.2,#1)--(-.2,#1) node[left]{$#2$};}
\newcommand{\nycoord}[2]{
	\draw (-.2,#1)--(.2,#1) node[right]{$#2$};}
\newcommand{\dlim}{\displaystyle\lim}
\newcommand{\dlimx}[1]{\displaystyle\lim_{x \rightarrow #1}}
\newcommand{\stickfig}[2]{
	\draw (#1,#2) arc(-90:270:2mm);
	\draw (#1,#2)--(#1,#2-.5) (#1-.25,#2-.75)--(#1,#2-.5)--(#1+.25,#2-.75) (#1-.2,#2-.2)--(#1+.2,#2-.2);}	

%\newcounter{example}
%\setcounter{example}{1}
%\newcounter{preFrameExample}
%\AtBeginEnvironment{frame}{\setcounter{preFrameExample}{\value{example}}}
%\newcommand{\ex}[1]{
%	 \setcounter{example}{\value{preFrameExample}}
%	 \textcolor{green}{\small\fbox{Example \arabic{example}: #1}}\\[8pt]
%	\stepcounter{example}}
%\newcommand{\exans}[1]{
%	\SUBTRACT{\value{preFrameExample}}{1}{\n}
%	 \textcolor{green}{\small\fbox{Solution \n: #1}}\\[8pt]}
\mode<presentation> {

% The Beamer class comes with a number of default slide themes
% which change the colors and layouts of slides. Below this is a list
% of all the themes, uncomment each in turn to see what they look like.


\usetheme{CambridgeUS}
\usecolortheme[named=black]{structure}


\newcommand{\studentcolor}[0]{ForestGreen}
\newcommand{\normcolor}[0]{NavyBlue}
\newcommand{\alertcolor}{Red}

\setbeamercolor{normal text}{fg=\normcolor}
\setbeamercolor{frametitle}{fg=\normcolor}
\setbeamercolor{section in head/foot}{fg=Black, bg=Gray!20}
\setbeamercolor{subsection in head/foot}{fg=Green!70!Black, bg=Gray!10}
\setbeamercolor{alerted text}{fg=\alertcolor}
\setbeamerfont{alerted text}{series=\bf}
\setbeamertemplate{enumerate items}[default]
\setbeamercolor{enumerate item}{fg=\normcolor}

\setbeamertemplate{footline} % To remove the footer line in all slides uncomment this line
%\setbeamertemplate{footline}[page number] % To replace the footer line in all slides with a simple slide count uncomment this line

\setbeamertemplate{navigation symbols}{} % To remove the navigation symbols from the bottom of all slides uncomment this line
}

\newcommand{\alertbox}[1]{\tcbox[on line, colframe=\alertcolor, colback=White, left=2pt,right=2pt,top=2pt,bottom=2pt]{\usebeamercolor*{normal text}#1}}


\newcommand{\startstu}{\setbeamercolor{normal text}{fg=\studentcolor}\usebeamercolor*{normal text}\setbeamercolor{enumerate item}{fg=\studentcolor}\usebeamercolor*{enumerate item}}
\newcommand{\stopstu}{\setbeamercolor{normal text}{fg=\normcolor}\usebeamercolor*{normal text}\setbeamercolor{enumerate item}{fg=\normcolor}\usebeamercolor*{enumerate item}}

\newcommand{\takenote}[1]{ \begin{collect}{notes}{}{}{}{}  #1  \end{collect}  \addtocounter{notestaken}{1}} %\ifthenelse{\value{notestaken}>0}{\hrulefill\\}{}

\makeatletter
\newcommand{\cover}{\alt{\beamer@makecovered}{\beamer@fakeinvisible}}
\newcommand{\ucover}[1]{\iftoggle{full}{}{\beamer@endcovered}\stopstu#1\startstu\iftoggle{full}{}{\beamer@startcovered}}
\makeatother

\newcommand{\skippause}{ \addtocounter{beamerpauses}{-1}}
\newcommand{\blockpres}{ \skippause \pause }

\newcommand{\studentify}[1]{\startstu #1  \stopstu }
\newcommand{\student}[1]{\iftoggle{full}{ \pause  \studentify{#1} }{\iftoggle{covered}{\studentify{#1}}{\cover{  #1 }}}}
\newcommand{\cstudent}[1]{\student{\begin{center} #1 \end{center}}}
\newcommand{\fullonly}[1]{\iftoggle{full}{ #1}{}}
\newcommand{\presentonly}[1]{\iftoggle{presentable}{ #1}{}}

\usepackage{xparse}
\usepackage{xifthen}

% shortcuts for commonly-used presentation elements
%\NewDocumentCommand{\slide}{o m}
% {\IfValueTF{#1}{\begin{frame}[t]{#1}}{\begin{frame}[t]} #2 \end{frame}}

\newtoggle{iscovered}

\newcommand{\slide}[2][]{%
%\setcounter{notestaken}{0}
\takenote{#2} 
%\ifthenelse{\equal{#1}{}}{\begin{frame}[t]}{\begin{frame}[t]{#1}} #2 \ifthenelse{\value{notestaken}>0}{ \note{\includecollection{notes}}}{} \end{frame}%
\ifthenelse{\equal{#1}{}}{\begin{frame}[t]}{\begin{frame}[t]{#1}} #2 \iftoggle{covered}{\settoggle{iscovered}{true}}{\settoggle{iscovered}{false}}  \note{ \iftoggle{iscovered}{}{\settoggle{covered}{true}} #2 \iftoggle{iscovered}{}{\settoggle{covered}{false}} } \end{frame}%
%\setcounter{notestaken}{0}
}
\newcommand{\defn}[2][]{%
 \setcounter{listcounter}{0}%
\ifthenelse{\equal{#1}{}}{\begin{block}{Definition}}{\begin{block}{#1 :}}%
 #2 \vspace{0.25em} \ifthenelse{\value{listcounter}>0}{\skippause}{} \pause \end{block}%
}



\newcommand{\arr}[2]{\begin{array}{#1}#2\end{array}}
\newcommand{\mat}[2]{\left[\arr{#1}{#2}\right]}
\newcommand{\carray}[1]{\arr{c}{#1}}
\newcommand{\larray}[1]{\arr{l}{#1}}
\newcommand{\rarray}[1]{\arr{r}{#1}}
\newcommand{\colvec}[1]{\mat{c}{#1}}

\newcommand{\itmz}[1]{\addtocounter{listcounter}{1} \begin{itemize}#1 \end{itemize} }
\newcommand{\subitem}[1]{\addtocounter{listcounter}{1} \begin{itemize} \item #1 \end{itemize}}
%
\newcommand{\enum}[1]{\addtocounter{listcounter}{1} \begin{enumerate} #1  \end{enumerate}  }


\newcommand{\algnlbl}[1]{\begin{align}#1  \end{align}} 
\newcommand{\algn}[1]{\begin{align*}#1  \end{align*}} 
\newcommand{\lgn}[1]{ \action<+->{#1} }
\newcommand{\slgn}[1]{\iftoggle{full}{\action<+->{ \startstu #1 \stopstu}}{ \cover{ #1 } } \takenote{$#1$}}

\newcommand{\chckmrk}{\alert{\checkmark}}

\usepackage{pifont}
\newcommand{\xmark}{\alert{\text{\large \ding{55}}}}

\newcommand{\return}[0]{\raisebox{.5ex}{\rotatebox[origin=c]{180}{$\Lsh$}}}
\usepackage{pbox}
%\newcommand{\ex}[1]{\rotatebox[origin=c]{10}{\uline{ex}}:$\;$\pbox[t][][b]{0.9\linewidth}{#1}}
\newcommand{\ex}[1]{\uline{ex}:$\;$\pbox[t][][t]{0.9\linewidth}{#1}}
\newcommand{\eg}[1]{e.g.,$\;$\pbox[t][][t]{0.9\linewidth}{#1}}
\newcommand{\tikzplot}[8][]{%
\begin{tikzpicture}

\begin{scope}[]%
\clip(-#2,-#4) rectangle (#3,#5);%
#8%
\end{scope}%
\laxis{#2}{#3}{#4}{#5}{#6}{#7}%
#1
\end{tikzpicture}%
}


\newcommand{\cancelslide}[1]{%
\begingroup%
\setbeamertemplate{background canvas}{%
\begin{tikzpicture}[remember picture,overlay]%
\draw[line width=2pt,red!60!black] %
  (current page.north west) -- (current page.south east);%
\draw[line width=2pt,red!60!black] %
  (current page.south west) -- (current page.north east);%
\end{tikzpicture}}%
#1%
\endgroup%
}
\renewcommand{\CancelColor}{\color{red}}
\newcommand{\twocols}[3][0.5]{\begin{columns}\begin{column}{#1\textwidth}#2\end{column}\hspace{1em}\vrule{}\hspace{1em}\begin{column}{#1\textwidth}#3\end{column}\end{columns}}

\newcommand{\twomini}[5][1]{\calculatespace \begin{minipage}[t]{\columnwidth}\begin{minipage}[][#1\contentheight][t]{#2\columnwidth}#4\end{minipage}\hfill\begin{minipage}[][#1\contentheight][t]{#3\columnwidth}#5\end{minipage}\end{minipage}}

\newcommand{\threemini}[7][1]{\calculatespace \begin{minipage}[t]{\columnwidth}\begin{minipage}[][#1\contentheight][t]{#2\columnwidth}#5\end{minipage}\hfill\begin{minipage}[][#1\contentheight][t]{#4\columnwidth}#6\end{minipage}\hfill\begin{minipage}[][#1\contentheight][t]{#3\columnwidth}#7\end{minipage}\end{minipage}}


\newcounter{listcounter}
\setcounter{listcounter}{0}



\newif\ifsidebartheme
\sidebarthemetrue

\newdimen\contentheight
\newdimen\contentwidth
\newdimen\contentleft
\newdimen\contentbottom
\makeatletter
\newcommand*{\calculatespace}{%
\contentheight=\paperheight%
\ifx\beamer@frametitle\@empty%
    \setbox\@tempboxa=\box\voidb@x%
  \else%
    \setbox\@tempboxa=\vbox{%
      \vbox{}%
      {\parskip0pt\usebeamertemplate***{frametitle}}%
    }%
    \ifsidebartheme%
      \advance\contentheight by-1em%
    \fi%
  \fi%
\advance\contentheight by-\ht\@tempboxa%
\advance\contentheight by-\dp\@tempboxa%
\advance\contentheight by-\beamer@frametopskip%
\ifbeamer@plainframe%
\contentbottom=0pt%
\else%
\advance\contentheight by-\headheight%
\advance\contentheight by\headdp%
\advance\contentheight by-\footheight%
\advance\contentheight by4pt%
\contentbottom=\footheight%
\advance\contentbottom by-4pt%
\fi%
\contentwidth=\paperwidth%
\ifbeamer@plainframe%
\contentleft=0pt%
\else%
\advance\contentwidth by-\beamer@rightsidebar%
\advance\contentwidth by-\beamer@leftsidebar\relax%
\contentleft=\beamer@leftsidebar%
\fi%
}
\makeatother



\iftoggle{dualscreen}{\setbeameroption{show notes on second screen=right}}{}
\usetikzlibrary{arrows}

\begin{document}
\section{Lecture 13}
\subsection{Theory}

\slide[Recall:]{
 \[\dd{}{t}\vec{x} = \mathbf{A}(t) \vec{x} \qquad \text{with } \mathbf{A} \text{ an } n\times n \text{ matrix} \]
\vfill
We can find $n$ solutions $\vec{x}(t)=e^{\lambda t}\vec{v}$ by finding the eigenvalues, $\lambda$, and eigenvectors, $\vec{v}$, of the matrix $\mathbf{A}$.\vfill
\vfill
i.e., solving
   \[\det(\mathbf{A} - \lambda \mathbf{I}) = 0\quad\text{and}\quad (\mathbf{A} - \lambda \mathbf{I})\vec{v}= 0\]
}

\slide[Homogeneous Problem and Superposition]{\vspace{-1em}
Suppose $\vec{x}_1(t), \;\vec{x}_2(t),\;\dots, \;\vec{x}_n(t)$ all solve the homogeneous problem \[\dd{}{t}\vec{x} = \mathbf{A}(t) \vec{x} \]\vfill
Then \[\vec{x}(t) = c_1\vec{x}_1(t)+c_2\vec{x}_2(t)+\dots+c_n\vec{x}_n(t)\]
also solves the same homogeneous problem.
\vfill
\student{
\algn{\dd{}{t}\vec{x} &= \sum_{i=1}^nc_i\dd{}{t}\vec{x}_i= \sum_{i=1}^nc_i \mathbf{A}(t) \vec{x}_i \\& =  \mathbf{A}(t)  \sum_{i=1}^nc_i\vec{x}_i \\
&=\mathbf{A}(t)\vec{x}(t)}
So, the $\vec{x}(t)$ above is the general solution to the homogeneous problem.
}
}
\subsection{Examples}
\slide[Find the general solution to \hfill \small $\larray{ \dd{x}{t} = -3x - 2y \\ \dd{y}{t} = -2x - 6y}$]{\student{\vspace{-3em}
\algn{\vec{x}(t)&=c_1\vec{x}_1(t) + c_2 \vec{x}_2(t)\\
&=c_1\underbrace{\underbrace{\mat{c}{-2\\1}}_{\text{eigenvector}}e^{-2t}}_{\text{eigensolution}} + c_2 \mat{c}{1\\2}e^{-7t}
}
Notes:
\itmz{\item The two eigenvectors are linearly independent
\subitem{$\Rightarrow$ The two eigensolutions/eigenmodes are linearly independent}
\item The solution has components along the two eigenvectors/eigendirection
\item The component along the second direction decays the fastest.
\item $c_1$ and $c_2$ are determined from initial conditions
}
}
}


\slide[Find the  solution to \hfill \small $\larray{ \dd{x}{t} = -3x - 2y \\ \dd{y}{t} = -2x - 6y}$ with $\larray{x(0)=5\\\\ y(0)=4}$]{\student{\vspace{-2em}
\algn{\ucover{\vec{x}(t)}&\ucover{=c_1\mat{c}{-2\\1}e^{-2t} + c_2 \mat{c}{1\\2}e^{-7t}}\\
\vec{x}(0)&=c_1\mat{c}{-2\\1} + c_2 \mat{c}{1\\2}  = \mat{c}{5\\4}\\
&\mat{cc|c}{-2c_1 & c_2 & 5\\c_1 &2c_2&4} \intertext{$2R_2+R_1\rightarrow R_1$ }
&\mat{cc|c}{0 & 5c_2 & 13\\c_1 &2c_2&4} 
 \intertext{$-\frac25R_1+R_2\rightarrow R_2$ }
&\mat{cc|c}{0 & 5c_2 & 13\\c_1 &0&-\frac{6}{5}}&\larray{c_1=-\frac65\\c_2=\frac{13}{5}} 
}
}
}

\slide[Initial Conditions and Eigenvectors]{
For first order systems of the form $\dd{}{t}\vec{x}=\mathbf{A}\vec{x}$, we have a general solution
 \[\vec{x}(t) = c_1\vec{x}_1(t)+c_2\vec{x}_2(t)+\dots+c_n\vec{x}_n(t),\] 
where with real and distinct eigenvalue of $\mathbf{A}$, we have: $\vec{x}_i(t)=e^{\lambda_i t}\vec{v}_i$
\student{
\enum{\vfill
\item The arbitrary coefficients $c_i$ come from decomposing the initial condition $\vec{x}(0)$ onto the basis of eigenvectors $\{\vec{v}_i\}$.\vfill
\item This decomposes the solution into $n$ different "eigenmodes".\vfill\subitem{$c_i$ is the initial "amplitude" of the $i^{th}$ eigenmode}\vfill
\item Each eigenmode grows ($\lambda_i>0$) or shrinks ($\lambda_i<0$) over time.
}
}
}

\slide[Find the general solution to \hfill \small $\larray{ \dd{x}{t} = - 2y \\ \dd{y}{t} = -2x - 3y}$]{\vspace{-1.5em}
\student{\algn{\dd{}{t} \mat{c}{x\\y}  &= \mat{cc}{0&-2\\-2&-3} \mat{c}{x\\y} \\ \det\left(  \mat{cc}{-\lambda&-2\\-2&-3-\lambda} \right) &= 0\\
-\lambda (-3-\lambda) - 4 &=0\\
\lambda^2 +3\lambda  - 4 &= 0\\
(\lambda-1)(\lambda+4) &=0\\
\lambda_{1,2} &=1, -4}
}
}

\slide[Find the general solution to \hfill \small $\larray{ \dd{x}{t} = - 2y \\ \dd{y}{t} = -2x - 3y}$]{\student{\vspace{-2em}
\algn{\uline{\lambda_1=1:} \quad &  \mat{cc|c}{0-1&-2 &0\\-2&-3-1&0} &&   \mat{cc|c}{-1&-2 &0\\-2&-1&0}
\intertext{row 2 and row 1 are linearly dependent: $R_2-2R_1\rightarrow R_2$}
& \mat{cc|c}{-1&-2 &0\\0&0&0} &-1x-2y&=0\\
&&x&=-2y\\
\vec{v}_1&=\mat{c}{-2\\1}&\vec{x}_1(t)&=\mat{c}{-2\\1}e^{t}
}
}
}

\slide[Find the general solution to \hfill \small $\larray{ \dd{x}{t} = - 2y \\ \dd{y}{t} = -2x - 3y}$]{\student{\vspace{-2em}
\algn{\uline{\lambda_2=-4:} \quad &    \mat{cc|c}{4&-2 &0\\-2&1&0}
\intertext{row 2 and row 1 are linearly dependent: $R_2+R_1/2\rightarrow R_2$}
& \mat{cc|c}{4&-2 &0\\0&0&0} &4x-2y&=0\\
&&2x&=y\\
\vec{v}_2&=\mat{c}{1\\2}&\vec{x}_2(t)&=\mat{c}{1\\2}e^{-4t}\\\\
\vec{x}(t) &= c_1\mat{c}{-2\\1}e^{t}+c_2\mat{c}{1\\2}e^{-4t}
}
}
}


\slide[Find the eigenvalues for the ODE:  \hfill \small $\larray{ \dd{x}{t} = -x + 2y \\ \dd{y}{t} = -2x - y}$]{\vspace{-1.5em}
\student{\algn{\mathbf{A}  &= \mat{cc}{-1&2\\-2&-1} \\
\det\left(  \mat{cc}{-1-\lambda&-2\\2&-1-\lambda} \right) &= 0\intertext{Characteristic equation}
 (-1-\lambda)^2 + 4 &=0\\
\lambda^2 +2\lambda  + 5 &= 0\\
\lambda_{1,2} &= \frac{-2\pm \sqrt{4-20}}{2} =\frac{-2\pm \sqrt{-16}}{2}\\
 \lambda_{1,2} &=  -1 \pm 2 i }
}
}

\slide[Complex conjugate eigenvalues pairs: $\lambda_{1,2}=r\pm i\omega $]{\vspace{-.5em}
\noindent Associated eigenvectors are also complex conjugates
\student{\[\vec{v}_{1,2}=\underbrace{\vec{a}}_{\text{real part}} \pm \underbrace{i\vec{b}}_{\text{imaginary part}}\]
\vfill
Proof:\vfill
Suppose $\vec{v}_{1}=\vec{a}+i\vec{b}$ with $\lambda_1=r+i \omega$
\algn{\mathbf{A} (\vec{a}+i\vec{b})  &= (r+i\omega) (\vec{a}+i\vec{b})\intertext{Take complex conjugate of both sides}
\mathbf{A} (\vec{a}-i\vec{b}) &=  (r-i\omega) (\vec{a}-i\vec{b})\\
\mathbf{A} \vec{v}_2 &=\lambda_2  \vec{v}_2  }
}
}

\slide[Find the eigenvectors for the ODE:  \hfill \small $\larray{ \dd{x}{t} = -x + 2y \\ \dd{y}{t} = -2x - y}$]{\vspace{-1.5em}
\student{\[ \lambda_{1,2} =  -1 \pm 2 i\]\vspace{-3em}
\algn{ \intertext{\uline{$\lambda_{1} =  -1 + 2 i$}} \mat{cc|c}{-1-(-1+2i)&2&0\\-2&-1-(-1+2i)&0} & \quad \rightarrow &\mat{cc|c}{-2i &2&0\\-2&-2i&0} \\
\intertext{$R_2-iR_1\rightarrow R_2$ and $\frac12 R_1 \rightarrow R_1$}
\mat{cc|c}{-i &1&0\\0&0&0} &&\rarray{ -i x + y = 0\\y=ix}\\\\
\vec{v}_1=\mat{c}{1\\i} = \mat{c}{1\\0}+i\mat{c}{0\\1}&  & \Rightarrow \vec{v}_2=\mat{c}{1\\-i} }
}
}

\slide[Find the general solution for the ODE:  \hfill \small $\larray{ \dd{x}{t} = -x + 2y \\ \dd{y}{t} = -2x - y}$]{\vspace{-1.5em}
\student{\[ \lambda_{1,2} =  -1 \pm 2 i \qquad \vec{v}_{1,2}=\mat{c}{1\\\pm i}\] 
\algn{\vec{x}(t) &= c_1e^{(-1+2i)t}\mat{c}{1\\ i} + c_2e^{(-1-2i)t}\mat{c}{1\\- i} \\\\
x(t)&=c_1 e^{(-1+2i)t} + c_2 e^{(-1-2i)t}\\
y(t)&=ic_1 e^{(-1+2i)t} - ic_2 e^{(-1-2i)t}
}\vfill
Can convert to purely real solution using Euler's identity (not covered).
}
}

\subsection{Summary}

\slide[General solution to the eigenproblem (2x2)]{
From linear algebra, we know that $\mathbf{A}\vec{v} = \lambda \vec{v}$ has a non-trivial solution $\vec{v}$ only if the determinant of the matrix $\mathbf{A}-\lambda \mathbf{I} $ is identically zero.
\vfill
That is, the constant $\lambda$ must satisfy the characteristic equation \[\det (\mathbf{A}-\lambda \mathbf{I})=0.\]\vfill
ex: $A=\mat{cc}{a&b\\c&d}$\student{$\det \left( \mat{cc}{a-\lambda &b\\c&d-\lambda} \right)=(a-\lambda)\cdot(d-\lambda)-bc=0$ \vfill
\algn{\underbrace{\lambda^2-(a+d)\lambda + ad - bc}_{\text{characterisitc poly.}} &=0 &\Rightarrow
\lambda = \frac{a+b\pm\sqrt{(a+d)^2-4(ad-bc)}}{2} }}
}

\slide[General solution to the eigenproblem (2x2)]{
From linear algebra, we know that $\mathbf{A}\vec{v} = \lambda \vec{v}$ has a non-trivial solution $\vec{v}$ only if the determinant of the matrix $\mathbf{A}-\lambda \mathbf{I} $ is identically zero.
\vfill
That is, the constant $\lambda$ must satisfy the characteristic equation \[\det (\mathbf{A}-\lambda \mathbf{I})=0.\]\vfill
\student{
Three possibilites
\enum{
\item $\mathbf{A}$ has 2 distinct real eigenvalues and eigenvectors\checkmark
\item The eigenvalues/vectors of  $\mathbf{A}$ occur as a complex conjugate pair \checkmark
\item One eigenvalue is repeated (not covered)
}}
}

\end{document}