\documentclass[11pt, dvipsnames, handout]{beamer}
\newtoggle{full}
\settoggle{full}{true}

\newtoggle{covered}
\settoggle{covered}{false}

\newtoggle{presentable}
\settoggle{presentable}{false}

\newtoggle{dualscreen}
\settoggle{dualscreen}{false}

\usepackage{pgfplots}
%\pgfplotsset{compat = newest}

\usepackage{pgfpages}

\setbeamertemplate{note page}{\pagecolor{yellow!5}\vfill \insertnote \vfill}
\usepackage{collect}
\definecollection{notes}
\newcounter{notestaken}

\usepackage{xpatch}

\usepackage{ulem}

\usepackage[framemethod=tikz]{mdframed}

\usepackage{scalerel}
\usepackage{calc}

%\usepackage{enumitem}
\setlength\fboxsep{.2em}

\usepackage{graphicx} % Allows including images
\usepackage{booktabs} % Allows the use of \toprule, \midrule and \bottomrule in tables

\xpatchcmd{\itemize}
  {\def\makelabel}
  {\setlength{\itemsep}{0.65 em}\def\makelabel}
  {}
  {}


\xpatchcmd{\beamer@enum@}
  {\def\makelabel}
  {\setlength{\itemsep}{0.65 em}\def\makelabel}
  {}
  {}


%\makeatletter
%\renewcommand{\itemize}[1][]{%
%  \beamer@ifempty{#1}{}{\def\beamer@defaultospec{#1}}%
%  \ifnum \@itemdepth >2\relax\@toodeep\else
%    \advance\@itemdepth\@ne
%    \beamer@computepref\@itemdepth% sets \beameritemnestingprefix
%    \usebeamerfont{itemize/enumerate \beameritemnestingprefix body}%
%    \usebeamercolor[fg]{itemize/enumerate \beameritemnestingprefix body}%
%    \usebeamertemplate{itemize/enumerate \beameritemnestingprefix body begin}%
%    \list
%      {\usebeamertemplate{itemize \beameritemnestingprefix item}}
%      {%
%        \setlength\topsep{1em}%NEW
%        \setlength\partopsep{1em}%NEW
%        \setlength\itemsep{1em}%NEW
%        \def\makelabel##1{%
%          {%
%            \hss\llap{{%
%                \usebeamerfont*{itemize \beameritemnestingprefix item}%
%                \usebeamercolor[fg]{itemize \beameritemnestingprefix item}##1}}%
%          }%
%        }%
%      }
%  \fi%
%  \beamer@cramped%
%  \raggedright%
%  \beamer@firstlineitemizeunskip%
%}
%
%
%
%
%
%\makeatother

%\setlist[beamer@enum@]{topsep=1 em}
%\let\origcheckmark\checkmark %screw you dingbat
%\let\checkmark\undefined %screw you dingbat
%\usepackage{dingbat} 
%\let\checkmark\origcheckmark %screw you dingbat






%\usepackage{fontawesome}

\usepackage{mathtools}
\usepackage{etoolbox, calculator}

\usepackage{xcolor}
\usepackage{tikz}
\usetikzlibrary{arrows.meta}
\usetikzlibrary{calc}
\usepackage[nomessages]{fp}
\usepackage{transparent}
\usepackage{accsupp}
%\usepackage{color, xcolor}

%colorblind-friendly palette
%\definecolor{dblue}{RGB}{51,34,136}
\definecolor{lblue}{RGB}{136,204,238}
%\definecolor{green}{RGB}{17,119,51}
\definecolor{tan}{RGB}{221,204,119}
%\definecolor{mauve}{RGB}{204,102,119}

\usepackage{tcolorbox}



\usepackage{xifthen}
\usepackage{nicefrac}
\usepackage{amsmath}
\usepackage{amsthm}
\usepackage{amssymb}
\theoremstyle{definition}
\newtheorem*{define}{Definition}
\newtheorem*{recall}{Recall}


\DeclareMathOperator{\tr}{tr}

\usepackage{multicol}
%\setlength{\columnsep}{1cm}

\usepackage{tablists, amsmath,vwcol, cancel, polynom}
\usetikzlibrary{shapes, patterns, decorations.shapes}
%\usepackage{tikzpeople}
\tikzstyle{vertex}=[shape=circle, minimum size=2mm, inner sep=0, fill]
\tikzstyle{opendot}=[shape=circle, minimum size=2mm, inner sep=0, fill=white, draw]

% common math quick commands
\newcommand{\nicedd}[2]{\nicefrac{\text{d}#1}{\text{d}#2}}
\newcommand{\dd}[2]{\dfrac{\text{d}#1}{\text{d}#2}}
\newcommand{\pd}[2]{\dfrac{\partial #1}{\partial#2}}
\renewcommand{\d}[1]{\text{d}#1}
\newcommand{\ddn}[3]{\dfrac{\text{d}^{#3}#1}{\text{d}#2^{#3}}}
\newcommand{\pdn}[3]{\dfrac{\partial^{#3}#1}{\partial#2^{#3}}}
\newcommand{\p}[0]{^{\prime}}
\newcommand{\pp}[0]{^{\prime\prime}}
\newcommand{\op}[2][\text{L}]{#1 \left[ #2 \right]}

\newcommand{\lap}[1]{\mathcal{L}\left\{#1\right\}}
\newcommand{\lapinv}[1]{\mathcal{L}^{-1}\left\{#1\right\}}
\newcommand{\lapint}[1]{\int_0^\infty e^{-st}#1dt}
\newcommand{\evalat}[2]{\Big|_{#1}^{#2}}

\newcommand{\paren}[1]{ \left( #1 \right)}

\newcommand{\haxis}[4][\normcolor]{\draw[#1, <->] (-#2,0)--(#3,0) node[right]{$#4$}; }


\newcommand{\axis}[4]{\draw[\normcolor, <->] (-#1,0)--(#2,0) 
node[right]{$x$};
\draw[help lines, <->] (0,-#3)--(0,#4) node[above]{$y$};}

\newcommand{\laxis}[6]{\draw[<->] (-#1,0)--(#2,0) 
node[right]{$#5$};
\draw[ <->] (0,-#3)--(0,#4) node[above]{$#6$};}
\newcommand{\xcoord}[2]{
	\draw (#1,.2)--(#1,-.2) node[below]{$#2$};}
\newcommand{\textnode}[3]{
	\draw (#1,#2) node[below]{$#3$};}
	
\newcommand{\nxcoord}[2]{
	\draw (#1,-.2)--(#1,.2) node[above]{$#2$};}
\newcommand{\ycoord}[2]{
	\draw (.2,#1)--(-.2,#1) node[left]{$#2$};}
\newcommand{\nycoord}[2]{
	\draw (-.2,#1)--(.2,#1) node[right]{$#2$};}
\newcommand{\dlim}{\displaystyle\lim}
\newcommand{\dlimx}[1]{\displaystyle\lim_{x \rightarrow #1}}
\newcommand{\stickfig}[2]{
	\draw (#1,#2) arc(-90:270:2mm);
	\draw (#1,#2)--(#1,#2-.5) (#1-.25,#2-.75)--(#1,#2-.5)--(#1+.25,#2-.75) (#1-.2,#2-.2)--(#1+.2,#2-.2);}	

%\newcounter{example}
%\setcounter{example}{1}
%\newcounter{preFrameExample}
%\AtBeginEnvironment{frame}{\setcounter{preFrameExample}{\value{example}}}
%\newcommand{\ex}[1]{
%	 \setcounter{example}{\value{preFrameExample}}
%	 \textcolor{green}{\small\fbox{Example \arabic{example}: #1}}\\[8pt]
%	\stepcounter{example}}
%\newcommand{\exans}[1]{
%	\SUBTRACT{\value{preFrameExample}}{1}{\n}
%	 \textcolor{green}{\small\fbox{Solution \n: #1}}\\[8pt]}
\mode<presentation> {

% The Beamer class comes with a number of default slide themes
% which change the colors and layouts of slides. Below this is a list
% of all the themes, uncomment each in turn to see what they look like.


\usetheme{CambridgeUS}
\usecolortheme[named=black]{structure}


\newcommand{\studentcolor}[0]{ForestGreen}
\newcommand{\normcolor}[0]{NavyBlue}
\newcommand{\alertcolor}{Red}

\setbeamercolor{normal text}{fg=\normcolor}
\setbeamercolor{frametitle}{fg=\normcolor}
\setbeamercolor{section in head/foot}{fg=Black, bg=Gray!20}
\setbeamercolor{subsection in head/foot}{fg=Green!70!Black, bg=Gray!10}
\setbeamercolor{alerted text}{fg=\alertcolor}
\setbeamerfont{alerted text}{series=\bf}
\setbeamertemplate{enumerate items}[default]
\setbeamercolor{enumerate item}{fg=\normcolor}

\setbeamertemplate{footline} % To remove the footer line in all slides uncomment this line
%\setbeamertemplate{footline}[page number] % To replace the footer line in all slides with a simple slide count uncomment this line

\setbeamertemplate{navigation symbols}{} % To remove the navigation symbols from the bottom of all slides uncomment this line
}

\newcommand{\alertbox}[1]{\tcbox[on line, colframe=\alertcolor, colback=White, left=2pt,right=2pt,top=2pt,bottom=2pt]{\usebeamercolor*{normal text}#1}}


\newcommand{\startstu}{\setbeamercolor{normal text}{fg=\studentcolor}\usebeamercolor*{normal text}\setbeamercolor{enumerate item}{fg=\studentcolor}\usebeamercolor*{enumerate item}}
\newcommand{\stopstu}{\setbeamercolor{normal text}{fg=\normcolor}\usebeamercolor*{normal text}\setbeamercolor{enumerate item}{fg=\normcolor}\usebeamercolor*{enumerate item}}

\newcommand{\takenote}[1]{ \begin{collect}{notes}{}{}{}{}  #1  \end{collect}  \addtocounter{notestaken}{1}} %\ifthenelse{\value{notestaken}>0}{\hrulefill\\}{}

\makeatletter
\newcommand{\cover}{\alt{\beamer@makecovered}{\beamer@fakeinvisible}}
\newcommand{\ucover}[1]{\iftoggle{full}{}{\beamer@endcovered}\stopstu#1\startstu\iftoggle{full}{}{\beamer@startcovered}}
\makeatother

\newcommand{\skippause}{ \addtocounter{beamerpauses}{-1}}
\newcommand{\blockpres}{ \skippause \pause }

\newcommand{\studentify}[1]{\startstu #1  \stopstu }
\newcommand{\student}[1]{\iftoggle{full}{ \pause  \studentify{#1} }{\iftoggle{covered}{\studentify{#1}}{\cover{  #1 }}}}
\newcommand{\cstudent}[1]{\student{\begin{center} #1 \end{center}}}
\newcommand{\fullonly}[1]{\iftoggle{full}{ #1}{}}
\newcommand{\presentonly}[1]{\iftoggle{presentable}{ #1}{}}

\usepackage{xparse}
\usepackage{xifthen}

% shortcuts for commonly-used presentation elements
%\NewDocumentCommand{\slide}{o m}
% {\IfValueTF{#1}{\begin{frame}[t]{#1}}{\begin{frame}[t]} #2 \end{frame}}

\newtoggle{iscovered}

\newcommand{\slide}[2][]{%
%\setcounter{notestaken}{0}
\takenote{#2} 
%\ifthenelse{\equal{#1}{}}{\begin{frame}[t]}{\begin{frame}[t]{#1}} #2 \ifthenelse{\value{notestaken}>0}{ \note{\includecollection{notes}}}{} \end{frame}%
\ifthenelse{\equal{#1}{}}{\begin{frame}[t]}{\begin{frame}[t]{#1}} #2 \iftoggle{covered}{\settoggle{iscovered}{true}}{\settoggle{iscovered}{false}}  \note{ \iftoggle{iscovered}{}{\settoggle{covered}{true}} #2 \iftoggle{iscovered}{}{\settoggle{covered}{false}} } \end{frame}%
%\setcounter{notestaken}{0}
}
\newcommand{\defn}[2][]{%
 \setcounter{listcounter}{0}%
\ifthenelse{\equal{#1}{}}{\begin{block}{Definition}}{\begin{block}{#1 :}}%
 #2 \vspace{0.25em} \ifthenelse{\value{listcounter}>0}{\skippause}{} \pause \end{block}%
}



\newcommand{\arr}[2]{\begin{array}{#1}#2\end{array}}
\newcommand{\mat}[2]{\left[\arr{#1}{#2}\right]}
\newcommand{\carray}[1]{\arr{c}{#1}}
\newcommand{\larray}[1]{\arr{l}{#1}}
\newcommand{\rarray}[1]{\arr{r}{#1}}
\newcommand{\colvec}[1]{\mat{c}{#1}}

\newcommand{\itmz}[1]{\addtocounter{listcounter}{1} \begin{itemize}#1 \end{itemize} }
\newcommand{\subitem}[1]{\addtocounter{listcounter}{1} \begin{itemize} \item #1 \end{itemize}}
%
\newcommand{\enum}[1]{\addtocounter{listcounter}{1} \begin{enumerate} #1  \end{enumerate}  }


\newcommand{\algnlbl}[1]{\begin{align}#1  \end{align}} 
\newcommand{\algn}[1]{\begin{align*}#1  \end{align*}} 
\newcommand{\lgn}[1]{ \action<+->{#1} }
\newcommand{\slgn}[1]{\iftoggle{full}{\action<+->{ \startstu #1 \stopstu}}{ \cover{ #1 } } \takenote{$#1$}}

\newcommand{\chckmrk}{\alert{\checkmark}}

\usepackage{pifont}
\newcommand{\xmark}{\alert{\text{\large \ding{55}}}}

\newcommand{\return}[0]{\raisebox{.5ex}{\rotatebox[origin=c]{180}{$\Lsh$}}}
\usepackage{pbox}
%\newcommand{\ex}[1]{\rotatebox[origin=c]{10}{\uline{ex}}:$\;$\pbox[t][][b]{0.9\linewidth}{#1}}
\newcommand{\ex}[1]{\uline{ex}:$\;$\pbox[t][][t]{0.9\linewidth}{#1}}
\newcommand{\eg}[1]{e.g.,$\;$\pbox[t][][t]{0.9\linewidth}{#1}}
\newcommand{\tikzplot}[8][]{%
\begin{tikzpicture}

\begin{scope}[]%
\clip(-#2,-#4) rectangle (#3,#5);%
#8%
\end{scope}%
\laxis{#2}{#3}{#4}{#5}{#6}{#7}%
#1
\end{tikzpicture}%
}


\newcommand{\cancelslide}[1]{%
\begingroup%
\setbeamertemplate{background canvas}{%
\begin{tikzpicture}[remember picture,overlay]%
\draw[line width=2pt,red!60!black] %
  (current page.north west) -- (current page.south east);%
\draw[line width=2pt,red!60!black] %
  (current page.south west) -- (current page.north east);%
\end{tikzpicture}}%
#1%
\endgroup%
}
\renewcommand{\CancelColor}{\color{red}}
\newcommand{\twocols}[3][0.5]{\begin{columns}\begin{column}{#1\textwidth}#2\end{column}\hspace{1em}\vrule{}\hspace{1em}\begin{column}{#1\textwidth}#3\end{column}\end{columns}}

\newcommand{\twomini}[5][1]{\calculatespace \begin{minipage}[t]{\columnwidth}\begin{minipage}[][#1\contentheight][t]{#2\columnwidth}#4\end{minipage}\hfill\begin{minipage}[][#1\contentheight][t]{#3\columnwidth}#5\end{minipage}\end{minipage}}

\newcommand{\threemini}[7][1]{\calculatespace \begin{minipage}[t]{\columnwidth}\begin{minipage}[][#1\contentheight][t]{#2\columnwidth}#5\end{minipage}\hfill\begin{minipage}[][#1\contentheight][t]{#4\columnwidth}#6\end{minipage}\hfill\begin{minipage}[][#1\contentheight][t]{#3\columnwidth}#7\end{minipage}\end{minipage}}


\newcounter{listcounter}
\setcounter{listcounter}{0}



\newif\ifsidebartheme
\sidebarthemetrue

\newdimen\contentheight
\newdimen\contentwidth
\newdimen\contentleft
\newdimen\contentbottom
\makeatletter
\newcommand*{\calculatespace}{%
\contentheight=\paperheight%
\ifx\beamer@frametitle\@empty%
    \setbox\@tempboxa=\box\voidb@x%
  \else%
    \setbox\@tempboxa=\vbox{%
      \vbox{}%
      {\parskip0pt\usebeamertemplate***{frametitle}}%
    }%
    \ifsidebartheme%
      \advance\contentheight by-1em%
    \fi%
  \fi%
\advance\contentheight by-\ht\@tempboxa%
\advance\contentheight by-\dp\@tempboxa%
\advance\contentheight by-\beamer@frametopskip%
\ifbeamer@plainframe%
\contentbottom=0pt%
\else%
\advance\contentheight by-\headheight%
\advance\contentheight by\headdp%
\advance\contentheight by-\footheight%
\advance\contentheight by4pt%
\contentbottom=\footheight%
\advance\contentbottom by-4pt%
\fi%
\contentwidth=\paperwidth%
\ifbeamer@plainframe%
\contentleft=0pt%
\else%
\advance\contentwidth by-\beamer@rightsidebar%
\advance\contentwidth by-\beamer@leftsidebar\relax%
\contentleft=\beamer@leftsidebar%
\fi%
}
\makeatother


\iftoggle{dualscreen}{\setbeameroption{show notes on second screen=right}}{}

\usepackage{gensymb}
\begin{document}

\section{Lecture 28}
\subsection{D'Alembert}
\slide[D'Alembert's Method: $y_tt=c^2y+xx$]{
The Separation of Variables/Fourier method can be used to analyze the spectrum of frequencies present in the wave.
\vfill
\centerline{Useful to understand music - not useful to visualize a wave.}
\vfill
\hrule
\vfill
D'Alembert's Method: an alternative approach based on change of coordinates:
\[\xi = x - ct,\qquad \eta = x+ct\]
\vfill
Basic idea, solutions are waves travelling with speed $\pm c$. 

\vfill


\centerline{\tikzplot[\xcoord{9.4247}{L}]{0.1}{9.5}{.02}{1.4}{x}{y}{

\draw[smooth, ultra  thick, red, domain=0:3.1415, samples=50] plot ({\x*3},{exp(-((\x-0.75)^2)/.06)}) node[xshift=-6cm, yshift=.65cm]{$c \rightarrow$};
}
}

}

\slide[D'Alembert's Method]{
\[\xi = x - ct,\qquad \eta = x+ct\]\vfill
Let's compute $\pdn{}{x}{2}$ in this new coordinate system.\vfill
\student{
Apply the chain rule:
\algn{\pd{}{x} &= \cancelto{1}{\pd{\xi}{x}}\pd{}{\eta} + \cancelto{1}{\pd{\eta}{x}}\pd{}{\eta} = \pd{}{\xi}+\pd{}{\eta}\\
\pdn{}{x}{2} = \pd{}{x} \pd{}{x} &= \left(\pd{}{\xi}+\pd{}{\eta}\right) \left(\pd{}{\xi}+\pd{}{\eta}\right) \\
&=\pdn{}{\xi}{2} + 2\frac{\partial^2}{\partial\xi\partial\eta} + \pdn{}{\eta}{2}  }
}
}

\slide[D'Alembert's Method]{
\[\xi = x - ct,\qquad \eta = x+ct\]\vfill
Let's compute $\pdn{}{t}{2}$ in this new coordinate system.\vfill
\student{
Apply the chain rule:
\algn{\pd{}{t} &= \cancelto{-c}{\pd{\xi}{t}}\pd{}{\eta} + \cancelto{+c}{\pd{\eta}{x}}\pd{}{\eta} = -c\pd{}{\xi}+c\pd{}{\eta}\\
\pdn{}{t}{2} = \pd{}{t} \pd{}{t} &= \left(-c\pd{}{\xi}+c\pd{}{\eta}\right) \left(-c\pd{}{\xi}+c\pd{}{\eta}\right) \\
&=c^2\left( \pdn{}{\xi}{2} - 2\frac{\partial^2}{\partial\xi\partial\eta} + \pdn{}{\eta}{2} \right)  }
}
}

\slide[D'Alembert's Method]{
\[ \pdn{}{x}{2}  = \pdn{}{\xi}{2} + 2\frac{\partial^2}{\partial\xi\partial\eta} + \pdn{}{\eta}{2} \qquad \pdn{}{t}{2} = c^2\pdn{}{\xi}{2} - 2c^2\frac{\partial^2}{\partial\xi\partial\eta} + c^2\pdn{}{\eta}{2} \]\vfill
Write down the wave equation in the new coordinate system.\vfill
\student{

\algn{  y_{tt} &= c^2 y_{xx} \\
    c^2\pdn{y}{\xi}{2} - 2c^2\frac{\partial^2y}{\partial\xi\partial\eta} + c^2\pdn{y}{\eta}{2}&= c^2\pdn{y}{\xi}{2} + c^2 2\frac{\partial^2y}{\partial\xi\partial\eta} + c^2\pdn{y}{\eta}{2}\\
0&=4c^2 \frac{\partial^2y}{\partial\xi\partial\eta}
}
}
}

\slide[D'Alembert's Method]{
\[0= \frac{\partial^2y}{\partial\xi\partial\eta}\]
Find the general solution to the wave equation:\vfill
\student{Drop the multiplicative constants, integrate w.r.t. $\xi$
\algn{y_\eta(\xi,\eta) &= \int \frac{\partial^2y}{\partial\xi\partial\eta} d\xi =   \int 0 d\xi = C(\eta)\\
y(\xi,\eta) &= \int C(\eta) d\eta = A(\eta) + B(\xi)}
where $A$ and $B$ are single variable functions.\[y(x,t)=A(x-ct)+B(x+ct)\]
}
}

\slide[]{\vspace{-1.5em}\[y(x,t)=A(x-ct)+B(x+ct)\]
$A(x-ct)$: moves rightwards
\centerline{\tikzplot[\xcoord{9.4247}{L}]{0.1}{9.5}{.02}{1.4}{x}{y}{
\draw[smooth, ultra  thick, red, domain=0:3.1415, samples=50] plot ({\x*3},{exp(-((\x-0.75)^2)/.06)}) node[xshift=-7.5cm, yshift=1.2cm]{$c \rightarrow$};
\foreach \ct in {1,2,3,4,5}{
\draw[smooth, thick, red, domain=0:3.1415, samples=50] plot ({\x*3+\ct},{exp(-((\x-0.75)^2)/.06)});
\draw (3*.75+\ct,1.2) node[red]{t=\ct};
}
}
}
\vfill
$B(x+ct)$: moves leftwards
\centerline{\tikzplot[\xcoord{9.4247}{L}]{0.1}{9.5}{1}{1.4}{x}{y}{
\draw[smooth, ultra  thick, Plum, domain=0:3.1415, samples=50] plot ({\x*3},{8*(\x-2.5)*exp(-((\x-2.5)^2)/.06)}) node[xshift=-1cm, yshift=1.2cm]{$ \leftarrow -c $};
\foreach \ct in {1,2,3,4,5}{
\draw[smooth, thick, Plum, domain=0:3.1415, samples=50] plot ({\x*3-\ct},{8*(\x-2.5)*exp(-((\x-2.5)^2)/.06)});
\draw (3*2.75-\ct,1.2) node[Plum]{t=\ct};
}
}
}
\vfill
These are just two unrelated examples, the wave profiles $A$ and $B$ are not completely different like in this slide.
}

\slide[D'Alembert's Formula]{
\vspace{-1em}
General Solution to:
\vspace{-.5em}
\algn{y_{tt}&=c^2y_{xx} \quad 0<x<L & y(0)&=y(L)=0, \\   y(x,0)&= f(x)& y_t(x,0)&=g(x)}
\vfill
\[\boxed{y(x,t)=A(x-ct)+B(x+ct)}\]
\vfill
\algn{A(z) &= \frac12 \left[ F(z) - \frac1c\int_0^z G(x)dx\right]& B(z) &= \frac12 \left[ F(z) + \frac1c\int_0^z G(x)dx \right]}
\vfill
$F$ is the odd periodic extension of $f(x)$\\
$G$ is the odd periodic extension of $g(x)$

}


\slide[Intepreting D'Alembert's Formula]{\vspace{-3em}
\algn{y_{tt}&=c^2y_{xx} \quad 0<x<L & y(0)&=y(L)=0, \\   y(x,0)&= f(x)& y_t(x,0)&=g(x)}
\vfill
\[\boxed{y(x,t)=A(x-ct)+B(x+ct)}\]\vfill

Suppose $g(x)=0$, then $G(x)=0$ so \[A(x)=B(x)=\frac12 F(z)\]

We have odd periodic extensions of the initial condition moving left and right.\vfill This leads to reflection of the wave at the domain boundaries.

}

\slide[]{
$t=0$ - \textcolor{red}{$A$} and \textcolor{YellowOrange}{$B$} sum to give the \textcolor{black}{initial condition} \hfill \href{https://www.desmos.com/calculator/drbitic8pw}{[Desmos Version]}
\newcommand{\xzero}{1}
\centerline{
\newcommand{\ct}{0}
\tikz[scale=.6,domain=-5:5,samples=50]{
    \begin{axis}[axis on top=false, axis x line=middle, axis y line=middle, xmin=-5, xmax=5, ymin=-1.05, ymax=1.05,  width=20cm, height=6cm, xtick = {-5,0,5}, xticklabels = {-L,0,L}, ymajorticks=false]
        \draw (axis cs:2.25,0.5) node[red]{\Huge $c \rightarrow$};
       \draw (axis cs:0,0.5) node[YellowOrange]{\Huge $\leftarrow -c $};
        \addplot+[red, mark=none, smooth, line width=5pt,  samples=200, domain=-5:5] {0.5*exp(-(((x-\ct)-\xzero)^2)/0.02)-0.5*exp(-(((x-\ct)+\xzero)^2)/0.02)};
        \addplot+[YellowOrange, dashed,  mark=none, smooth, line width=5pt,  samples=200, domain=-5:5] {0.5*exp(-(((x+\ct)-\xzero)^2)/0.02)-0.5*exp(-(((x+\ct)+\xzero)^2)/0.02)};
        \addplot+[black, mark=none, line width=1pt, smooth, samples=200, domain=0:5] {0.5*exp(-(((x-\ct)-\xzero)^2)/0.02)-0.5*exp(-(((x-\ct)+\xzero)^2)/0.02)+0.5*exp(-(((x+\ct)-\xzero)^2)/0.02)-0.5*exp(-(((x+\ct)+\xzero)^2)/0.02)};

   \end{axis}
}
}
\vfill
$t=0.8$ - \textcolor{YellowOrange}{$B$} hits the boundary at $x=0$
\centerline{
\newcommand{\ct}{.8}
\tikz[scale=.6,domain=-5:5,samples=50]{
    \begin{axis}[axis on top=false, axis x line=middle, axis y line=middle, xmin=-5, xmax=5, ymin=-.65, ymax=.65,  width=20cm, height=3.5cm, xtick = {-5,0,5}, xticklabels = {-L,0,L}, ymajorticks=false]
        % plot first function
        \addplot+[red, mark=none, smooth, line width=5pt,  samples=200, domain=-5:5] {0.5*exp(-(((x-\ct)-\xzero)^2)/0.02)-0.5*exp(-((((x-\ct))+\xzero)^2)/0.02)};
        \addplot+[YellowOrange, dashed,  mark=none, smooth, line width=5pt,  samples=200, domain=-5:5] {0.5*exp(-(((x+\ct)-\xzero)^2)/0.02)-0.5*exp(-(((x+\ct)+\xzero)^2)/0.02)};
        \addplot+[black, mark=none, line width=1pt, smooth, samples=200, domain=0:5] {0.5*exp(-(((x-\ct)-\xzero)^2)/0.02)-0.5*exp(-(((x-\ct)+\xzero)^2)/0.02)+0.5*exp(-(((x+\ct)-\xzero)^2)/0.02)-0.5*exp(-(((x+\ct)+\xzero)^2)/0.02)};

   \end{axis}
}
}

\vfill
$t=1$ - primary wave from \textcolor{YellowOrange}{$B$} annihilates with the odd extension of  \textcolor{red}{$A$}
\centerline{
\newcommand{\ct}{1}
\tikz[scale=.6,domain=-5:5,samples=50]{
    \begin{axis}[axis on top=false, axis x line=middle, axis y line=middle, xmin=-5, xmax=5, ymin=-.65, ymax=.65,  width=20cm, height=3.5cm, xtick = {-5,0,5}, xticklabels = {-L,0,L}, ymajorticks=false]
        % plot first function
        \addplot+[red, mark=none, smooth, line width=5pt,  samples=200, domain=-5:5] {0.5*exp(-(((x-\ct)-\xzero)^2)/0.02)-0.5*exp(-((((x-\ct))+\xzero)^2)/0.02)};
        \addplot+[YellowOrange, dashed,  mark=none, smooth, line width=5pt,  samples=200, domain=-5:5] {0.5*exp(-(((x+\ct)-\xzero)^2)/0.02)-0.5*exp(-(((x+\ct)+\xzero)^2)/0.02)};
        \addplot+[black, mark=none, line width=1pt, smooth, samples=200, domain=0:5] {0.5*exp(-(((x-\ct)-\xzero)^2)/0.02)-0.5*exp(-(((x-\ct)+\xzero)^2)/0.02)+0.5*exp(-(((x+\ct)-\xzero)^2)/0.02)-0.5*exp(-(((x+\ct)+\xzero)^2)/0.02)};

   \end{axis}
}
}
\vfill
$t=1.8$ - wave is reflected after hitting boundary
\centerline{
\newcommand{\ct}{1.8}
\tikz[scale=.6,domain=-5:5,samples=50]{
    \begin{axis}[axis on top=false, axis x line=middle, axis y line=middle, xmin=-5, xmax=5, ymin=-.65, ymax=.65,  width=20cm, height=3.5cm, xtick = {-5,0,5}, xticklabels = {-L,0,L}, ymajorticks=false]
        % plot first function
        \addplot+[red, mark=none, smooth, line width=5pt,  samples=200, domain=-5:5] {0.5*exp(-(((x-\ct)-\xzero)^2)/0.02)-0.5*exp(-((((x-\ct))+\xzero)^2)/0.02)};
        \addplot+[YellowOrange, dashed,  mark=none, smooth, line width=5pt,  samples=200, domain=-5:5] {0.5*exp(-(((x+\ct)-\xzero)^2)/0.02)-0.5*exp(-(((x+\ct)+\xzero)^2)/0.02)};
        \addplot+[black, mark=none, line width=1pt, smooth, samples=200, domain=0:5] {0.5*exp(-(((x-\ct)-\xzero)^2)/0.02)-0.5*exp(-(((x-\ct)+\xzero)^2)/0.02)+0.5*exp(-(((x+\ct)-\xzero)^2)/0.02)-0.5*exp(-(((x+\ct)+\xzero)^2)/0.02)};

   \end{axis}
}
}


}

\end{document}