\documentclass[11pt, dvipsnames, handout]{beamer}
\newtoggle{full}
\settoggle{full}{true}

\newtoggle{covered}
\settoggle{covered}{false}

\newtoggle{presentable}
\settoggle{presentable}{false}

\newtoggle{dualscreen}
\settoggle{dualscreen}{false}

\usepackage{pgfplots}
%\pgfplotsset{compat = newest}

\usepackage{pgfpages}

\setbeamertemplate{note page}{\pagecolor{yellow!5}\vfill \insertnote \vfill}
\usepackage{collect}
\definecollection{notes}
\newcounter{notestaken}

\usepackage{xpatch}

\usepackage{ulem}

\usepackage[framemethod=tikz]{mdframed}

\usepackage{scalerel}
\usepackage{calc}

%\usepackage{enumitem}
\setlength\fboxsep{.2em}

\usepackage{graphicx} % Allows including images
\usepackage{booktabs} % Allows the use of \toprule, \midrule and \bottomrule in tables

\xpatchcmd{\itemize}
  {\def\makelabel}
  {\setlength{\itemsep}{0.65 em}\def\makelabel}
  {}
  {}


\xpatchcmd{\beamer@enum@}
  {\def\makelabel}
  {\setlength{\itemsep}{0.65 em}\def\makelabel}
  {}
  {}


%\makeatletter
%\renewcommand{\itemize}[1][]{%
%  \beamer@ifempty{#1}{}{\def\beamer@defaultospec{#1}}%
%  \ifnum \@itemdepth >2\relax\@toodeep\else
%    \advance\@itemdepth\@ne
%    \beamer@computepref\@itemdepth% sets \beameritemnestingprefix
%    \usebeamerfont{itemize/enumerate \beameritemnestingprefix body}%
%    \usebeamercolor[fg]{itemize/enumerate \beameritemnestingprefix body}%
%    \usebeamertemplate{itemize/enumerate \beameritemnestingprefix body begin}%
%    \list
%      {\usebeamertemplate{itemize \beameritemnestingprefix item}}
%      {%
%        \setlength\topsep{1em}%NEW
%        \setlength\partopsep{1em}%NEW
%        \setlength\itemsep{1em}%NEW
%        \def\makelabel##1{%
%          {%
%            \hss\llap{{%
%                \usebeamerfont*{itemize \beameritemnestingprefix item}%
%                \usebeamercolor[fg]{itemize \beameritemnestingprefix item}##1}}%
%          }%
%        }%
%      }
%  \fi%
%  \beamer@cramped%
%  \raggedright%
%  \beamer@firstlineitemizeunskip%
%}
%
%
%
%
%
%\makeatother

%\setlist[beamer@enum@]{topsep=1 em}
%\let\origcheckmark\checkmark %screw you dingbat
%\let\checkmark\undefined %screw you dingbat
%\usepackage{dingbat} 
%\let\checkmark\origcheckmark %screw you dingbat






%\usepackage{fontawesome}

\usepackage{mathtools}
\usepackage{etoolbox, calculator}

\usepackage{xcolor}
\usepackage{tikz}
\usetikzlibrary{arrows.meta}
\usetikzlibrary{calc}
\usepackage[nomessages]{fp}
\usepackage{transparent}
\usepackage{accsupp}
%\usepackage{color, xcolor}

%colorblind-friendly palette
%\definecolor{dblue}{RGB}{51,34,136}
\definecolor{lblue}{RGB}{136,204,238}
%\definecolor{green}{RGB}{17,119,51}
\definecolor{tan}{RGB}{221,204,119}
%\definecolor{mauve}{RGB}{204,102,119}

\usepackage{tcolorbox}



\usepackage{xifthen}
\usepackage{nicefrac}
\usepackage{amsmath}
\usepackage{amsthm}
\usepackage{amssymb}
\theoremstyle{definition}
\newtheorem*{define}{Definition}
\newtheorem*{recall}{Recall}


\DeclareMathOperator{\tr}{tr}

\usepackage{multicol}
%\setlength{\columnsep}{1cm}

\usepackage{tablists, amsmath,vwcol, cancel, polynom}
\usetikzlibrary{shapes, patterns, decorations.shapes}
%\usepackage{tikzpeople}
\tikzstyle{vertex}=[shape=circle, minimum size=2mm, inner sep=0, fill]
\tikzstyle{opendot}=[shape=circle, minimum size=2mm, inner sep=0, fill=white, draw]

% common math quick commands
\newcommand{\nicedd}[2]{\nicefrac{\text{d}#1}{\text{d}#2}}
\newcommand{\dd}[2]{\dfrac{\text{d}#1}{\text{d}#2}}
\newcommand{\pd}[2]{\dfrac{\partial #1}{\partial#2}}
\renewcommand{\d}[1]{\text{d}#1}
\newcommand{\ddn}[3]{\dfrac{\text{d}^{#3}#1}{\text{d}#2^{#3}}}
\newcommand{\pdn}[3]{\dfrac{\partial^{#3}#1}{\partial#2^{#3}}}
\newcommand{\p}[0]{^{\prime}}
\newcommand{\pp}[0]{^{\prime\prime}}
\newcommand{\op}[2][\text{L}]{#1 \left[ #2 \right]}

\newcommand{\lap}[1]{\mathcal{L}\left\{#1\right\}}
\newcommand{\lapinv}[1]{\mathcal{L}^{-1}\left\{#1\right\}}
\newcommand{\lapint}[1]{\int_0^\infty e^{-st}#1dt}
\newcommand{\evalat}[2]{\Big|_{#1}^{#2}}

\newcommand{\paren}[1]{ \left( #1 \right)}

\newcommand{\haxis}[4][\normcolor]{\draw[#1, <->] (-#2,0)--(#3,0) node[right]{$#4$}; }


\newcommand{\axis}[4]{\draw[\normcolor, <->] (-#1,0)--(#2,0) 
node[right]{$x$};
\draw[help lines, <->] (0,-#3)--(0,#4) node[above]{$y$};}

\newcommand{\laxis}[6]{\draw[<->] (-#1,0)--(#2,0) 
node[right]{$#5$};
\draw[ <->] (0,-#3)--(0,#4) node[above]{$#6$};}
\newcommand{\xcoord}[2]{
	\draw (#1,.2)--(#1,-.2) node[below]{$#2$};}
\newcommand{\textnode}[3]{
	\draw (#1,#2) node[below]{$#3$};}
	
\newcommand{\nxcoord}[2]{
	\draw (#1,-.2)--(#1,.2) node[above]{$#2$};}
\newcommand{\ycoord}[2]{
	\draw (.2,#1)--(-.2,#1) node[left]{$#2$};}
\newcommand{\nycoord}[2]{
	\draw (-.2,#1)--(.2,#1) node[right]{$#2$};}
\newcommand{\dlim}{\displaystyle\lim}
\newcommand{\dlimx}[1]{\displaystyle\lim_{x \rightarrow #1}}
\newcommand{\stickfig}[2]{
	\draw (#1,#2) arc(-90:270:2mm);
	\draw (#1,#2)--(#1,#2-.5) (#1-.25,#2-.75)--(#1,#2-.5)--(#1+.25,#2-.75) (#1-.2,#2-.2)--(#1+.2,#2-.2);}	

%\newcounter{example}
%\setcounter{example}{1}
%\newcounter{preFrameExample}
%\AtBeginEnvironment{frame}{\setcounter{preFrameExample}{\value{example}}}
%\newcommand{\ex}[1]{
%	 \setcounter{example}{\value{preFrameExample}}
%	 \textcolor{green}{\small\fbox{Example \arabic{example}: #1}}\\[8pt]
%	\stepcounter{example}}
%\newcommand{\exans}[1]{
%	\SUBTRACT{\value{preFrameExample}}{1}{\n}
%	 \textcolor{green}{\small\fbox{Solution \n: #1}}\\[8pt]}
\mode<presentation> {

% The Beamer class comes with a number of default slide themes
% which change the colors and layouts of slides. Below this is a list
% of all the themes, uncomment each in turn to see what they look like.


\usetheme{CambridgeUS}
\usecolortheme[named=black]{structure}


\newcommand{\studentcolor}[0]{ForestGreen}
\newcommand{\normcolor}[0]{NavyBlue}
\newcommand{\alertcolor}{Red}

\setbeamercolor{normal text}{fg=\normcolor}
\setbeamercolor{frametitle}{fg=\normcolor}
\setbeamercolor{section in head/foot}{fg=Black, bg=Gray!20}
\setbeamercolor{subsection in head/foot}{fg=Green!70!Black, bg=Gray!10}
\setbeamercolor{alerted text}{fg=\alertcolor}
\setbeamerfont{alerted text}{series=\bf}
\setbeamertemplate{enumerate items}[default]
\setbeamercolor{enumerate item}{fg=\normcolor}

\setbeamertemplate{footline} % To remove the footer line in all slides uncomment this line
%\setbeamertemplate{footline}[page number] % To replace the footer line in all slides with a simple slide count uncomment this line

\setbeamertemplate{navigation symbols}{} % To remove the navigation symbols from the bottom of all slides uncomment this line
}

\newcommand{\alertbox}[1]{\tcbox[on line, colframe=\alertcolor, colback=White, left=2pt,right=2pt,top=2pt,bottom=2pt]{\usebeamercolor*{normal text}#1}}


\newcommand{\startstu}{\setbeamercolor{normal text}{fg=\studentcolor}\usebeamercolor*{normal text}\setbeamercolor{enumerate item}{fg=\studentcolor}\usebeamercolor*{enumerate item}}
\newcommand{\stopstu}{\setbeamercolor{normal text}{fg=\normcolor}\usebeamercolor*{normal text}\setbeamercolor{enumerate item}{fg=\normcolor}\usebeamercolor*{enumerate item}}

\newcommand{\takenote}[1]{ \begin{collect}{notes}{}{}{}{}  #1  \end{collect}  \addtocounter{notestaken}{1}} %\ifthenelse{\value{notestaken}>0}{\hrulefill\\}{}

\makeatletter
\newcommand{\cover}{\alt{\beamer@makecovered}{\beamer@fakeinvisible}}
\newcommand{\ucover}[1]{\iftoggle{full}{}{\beamer@endcovered}\stopstu#1\startstu\iftoggle{full}{}{\beamer@startcovered}}
\makeatother

\newcommand{\skippause}{ \addtocounter{beamerpauses}{-1}}
\newcommand{\blockpres}{ \skippause \pause }

\newcommand{\studentify}[1]{\startstu #1  \stopstu }
\newcommand{\student}[1]{\iftoggle{full}{ \pause  \studentify{#1} }{\iftoggle{covered}{\studentify{#1}}{\cover{  #1 }}}}
\newcommand{\cstudent}[1]{\student{\begin{center} #1 \end{center}}}
\newcommand{\fullonly}[1]{\iftoggle{full}{ #1}{}}
\newcommand{\presentonly}[1]{\iftoggle{presentable}{ #1}{}}

\usepackage{xparse}
\usepackage{xifthen}

% shortcuts for commonly-used presentation elements
%\NewDocumentCommand{\slide}{o m}
% {\IfValueTF{#1}{\begin{frame}[t]{#1}}{\begin{frame}[t]} #2 \end{frame}}

\newtoggle{iscovered}

\newcommand{\slide}[2][]{%
%\setcounter{notestaken}{0}
\takenote{#2} 
%\ifthenelse{\equal{#1}{}}{\begin{frame}[t]}{\begin{frame}[t]{#1}} #2 \ifthenelse{\value{notestaken}>0}{ \note{\includecollection{notes}}}{} \end{frame}%
\ifthenelse{\equal{#1}{}}{\begin{frame}[t]}{\begin{frame}[t]{#1}} #2 \iftoggle{covered}{\settoggle{iscovered}{true}}{\settoggle{iscovered}{false}}  \note{ \iftoggle{iscovered}{}{\settoggle{covered}{true}} #2 \iftoggle{iscovered}{}{\settoggle{covered}{false}} } \end{frame}%
%\setcounter{notestaken}{0}
}
\newcommand{\defn}[2][]{%
 \setcounter{listcounter}{0}%
\ifthenelse{\equal{#1}{}}{\begin{block}{Definition}}{\begin{block}{#1 :}}%
 #2 \vspace{0.25em} \ifthenelse{\value{listcounter}>0}{\skippause}{} \pause \end{block}%
}



\newcommand{\arr}[2]{\begin{array}{#1}#2\end{array}}
\newcommand{\mat}[2]{\left[\arr{#1}{#2}\right]}
\newcommand{\carray}[1]{\arr{c}{#1}}
\newcommand{\larray}[1]{\arr{l}{#1}}
\newcommand{\rarray}[1]{\arr{r}{#1}}
\newcommand{\colvec}[1]{\mat{c}{#1}}

\newcommand{\itmz}[1]{\addtocounter{listcounter}{1} \begin{itemize}#1 \end{itemize} }
\newcommand{\subitem}[1]{\addtocounter{listcounter}{1} \begin{itemize} \item #1 \end{itemize}}
%
\newcommand{\enum}[1]{\addtocounter{listcounter}{1} \begin{enumerate} #1  \end{enumerate}  }


\newcommand{\algnlbl}[1]{\begin{align}#1  \end{align}} 
\newcommand{\algn}[1]{\begin{align*}#1  \end{align*}} 
\newcommand{\lgn}[1]{ \action<+->{#1} }
\newcommand{\slgn}[1]{\iftoggle{full}{\action<+->{ \startstu #1 \stopstu}}{ \cover{ #1 } } \takenote{$#1$}}

\newcommand{\chckmrk}{\alert{\checkmark}}

\usepackage{pifont}
\newcommand{\xmark}{\alert{\text{\large \ding{55}}}}

\newcommand{\return}[0]{\raisebox{.5ex}{\rotatebox[origin=c]{180}{$\Lsh$}}}
\usepackage{pbox}
%\newcommand{\ex}[1]{\rotatebox[origin=c]{10}{\uline{ex}}:$\;$\pbox[t][][b]{0.9\linewidth}{#1}}
\newcommand{\ex}[1]{\uline{ex}:$\;$\pbox[t][][t]{0.9\linewidth}{#1}}
\newcommand{\eg}[1]{e.g.,$\;$\pbox[t][][t]{0.9\linewidth}{#1}}
\newcommand{\tikzplot}[8][]{%
\begin{tikzpicture}

\begin{scope}[]%
\clip(-#2,-#4) rectangle (#3,#5);%
#8%
\end{scope}%
\laxis{#2}{#3}{#4}{#5}{#6}{#7}%
#1
\end{tikzpicture}%
}


\newcommand{\cancelslide}[1]{%
\begingroup%
\setbeamertemplate{background canvas}{%
\begin{tikzpicture}[remember picture,overlay]%
\draw[line width=2pt,red!60!black] %
  (current page.north west) -- (current page.south east);%
\draw[line width=2pt,red!60!black] %
  (current page.south west) -- (current page.north east);%
\end{tikzpicture}}%
#1%
\endgroup%
}
\renewcommand{\CancelColor}{\color{red}}
\newcommand{\twocols}[3][0.5]{\begin{columns}\begin{column}{#1\textwidth}#2\end{column}\hspace{1em}\vrule{}\hspace{1em}\begin{column}{#1\textwidth}#3\end{column}\end{columns}}

\newcommand{\twomini}[5][1]{\calculatespace \begin{minipage}[t]{\columnwidth}\begin{minipage}[][#1\contentheight][t]{#2\columnwidth}#4\end{minipage}\hfill\begin{minipage}[][#1\contentheight][t]{#3\columnwidth}#5\end{minipage}\end{minipage}}

\newcommand{\threemini}[7][1]{\calculatespace \begin{minipage}[t]{\columnwidth}\begin{minipage}[][#1\contentheight][t]{#2\columnwidth}#5\end{minipage}\hfill\begin{minipage}[][#1\contentheight][t]{#4\columnwidth}#6\end{minipage}\hfill\begin{minipage}[][#1\contentheight][t]{#3\columnwidth}#7\end{minipage}\end{minipage}}


\newcounter{listcounter}
\setcounter{listcounter}{0}



\newif\ifsidebartheme
\sidebarthemetrue

\newdimen\contentheight
\newdimen\contentwidth
\newdimen\contentleft
\newdimen\contentbottom
\makeatletter
\newcommand*{\calculatespace}{%
\contentheight=\paperheight%
\ifx\beamer@frametitle\@empty%
    \setbox\@tempboxa=\box\voidb@x%
  \else%
    \setbox\@tempboxa=\vbox{%
      \vbox{}%
      {\parskip0pt\usebeamertemplate***{frametitle}}%
    }%
    \ifsidebartheme%
      \advance\contentheight by-1em%
    \fi%
  \fi%
\advance\contentheight by-\ht\@tempboxa%
\advance\contentheight by-\dp\@tempboxa%
\advance\contentheight by-\beamer@frametopskip%
\ifbeamer@plainframe%
\contentbottom=0pt%
\else%
\advance\contentheight by-\headheight%
\advance\contentheight by\headdp%
\advance\contentheight by-\footheight%
\advance\contentheight by4pt%
\contentbottom=\footheight%
\advance\contentbottom by-4pt%
\fi%
\contentwidth=\paperwidth%
\ifbeamer@plainframe%
\contentleft=0pt%
\else%
\advance\contentwidth by-\beamer@rightsidebar%
\advance\contentwidth by-\beamer@leftsidebar\relax%
\contentleft=\beamer@leftsidebar%
\fi%
}
\makeatother

\usetikzlibrary{arrows}

\iftoggle{dualscreen}{\setbeameroption{show notes on second screen=right}}{}


\begin{document}
\section{Lecture 16}
\subsection{Review}
\slide{\centerline{
\begin{tikzpicture}
\draw[-left to, thick, ] (1,0) node[align=center, left]{time-domain} to [out=5,in=175]   node[label= above:$\mathcal{L}$]  {} (6,0) node[align=center, right]{s-domain};
\draw[-left to, thick]   (6,-0.1)  to [out=175,in=5]  node[label= below:$\mathcal{L}^{-1}$] {}(1,-0.1) ;
\draw (0.5,-4.25) ellipse (2cm and 3.8cm);
\draw (7,-4.25) ellipse (2cm and 3.8cm);
\draw[<->] (0.75,-1.5) node[left]{$f(t)$} to [out=7,in=173]  (6.5,-1.5) node[right]{$F(s)$} ;
\student{
\foreach \y\Y\i in
 { C/\frac{C}{s}/2, 
Ce^{-at}/\frac{C}{s+a}/3, 
t^n/\frac{n!}{s^{n+1}}/4,
\sin(\omega t)/\frac{\omega}{\omega^2+s^2}/5 ,
\cos(\omega t)/\frac{s}{\omega^2+s^2}/6 ,
\vdots\quad/\quad\vdots/7  }{
\draw[<->] (0.75,-\i-.5) node[left]{$\y$}  to [out=7,in=173]  (6.5,-\i-.5) node[right]{$\Y$} ;

}
}

\end{tikzpicture}}

}

\subsection{Standard Laplace Transforms}
\slide[Argument Scaling: $t \to \alpha t$ with constant $\alpha$]{
\student{\algn{\lap{f(\alpha t)}&=\lapint{f(\alpha t)} &\arr{l}{v=\alpha t\\du=\alpha dt}\\
&=\int_0^\infty e^{-\frac{s}{\alpha}u} f(u)\frac{du}{\alpha}\\
&=\frac1\alpha \underbrace{\int_0^\infty e^{-\frac{s}{\alpha}u} f(u)du}_{F\paren{\frac{s}{\alpha}}}\\
&=\frac{1}{\alpha} F\paren{\nicefrac{s}{\alpha}}   }
}}

\slide[First Shift Theorem: Multiplication by $e^{\alpha t}$]{
\student{
\algn{\lap{e^{\alpha t}f(t)}&=\lapint{e^{\alpha t}f(t)}\\
&=\int_0^\infty e^{-(s-\alpha)t}f(t)dt=F(s-\alpha)}
}\vfill
\ex{Suppose $Y(s)=\frac{1}{s+6}$, find $y(t)$.}
\student{\algn{ Y(s) &=  \underbrace{\frac{1}{s}}_{\lap{1}} \text{ with } s \to s+6 \\
y(t)& = e^{-6t} \lapinv{1/s}\\
y(t) &= e^{-6t}}
}\vfill
}

\slide{\ex{The LT of $\sin(4t)$ is $G(s)=\frac{4}{s^2+16}$.\\~\\What is the inverse of $F(s)=\frac{4}{s^2-6s+25}$?}
\student{
\algn{F(s)&=\frac{4}{\underbrace{s^2-6s+9}_{(s-3)^2} + 16} \\ 
&=\frac{4}{(s-3)^2+16}\\
&=\frac{4}{s^2+16} \text{ with } s\to s-3 \\\\
f(t)&=e^{3t}\lapinv{\frac{4}{s^2+16}}\\
&=e^{3t}\sin(4t) }
}
}

\slide[Resonance $\Leftrightarrow$ Differentiation in $s$-domain]{
\student{\algn{\lap{t^kf(t)} & = \lapint{t^kf(t)}& = \int_0^\infty \underbrace{e^{-st}t}_{-\dd{}{s}e^{-st}} t^{k-1} &f(t) dt\\
= -\dd{}{s} \int_0^\infty &e^{-st}t^{k-1} f(t) dt & \carray{\text{\tiny repeat same thing}\\ \dots \\ \text{\tiny k-1 more times}} = (-1)^k\ddn{}{s}{k}&F(s)
\intertext{with k=1}
\lap{tf(t)}&=-\dd{}{s} F(s) &ex: \lap{t\sin(\omega t)}=&\frac{2 \omega s}{(s^2+\omega^2)^2}\\
&& \lap{t\cos(\omega t)}=&\frac{s^2-\omega^2}{(s^2+\omega^2)^2}
}
}
}


\slide[The Heaviside Step Function: $u_c(t)$ or $u(t-c)$ or $H(t-c)$]
{
\student{Used to model effects that "turn-on" at some time $c$.}\vfill
\centerline{\tikzplot[\xcoord{3}{c}\ycoord{0}{0}\ycoord{2}{1}]{.1}{6}{.5}{2.5}{t}{u_c(t)}{
\draw[ultra thick, \studentcolor] (0,0)--(3,0);
\draw[ultra thick, \studentcolor] (3,2)--(6,2);
\draw[ultra thick, dashed, \studentcolor] (3,0)--(3,2);
\draw[\studentcolor] (3,0) node[vertex]{};
\draw[\studentcolor] (3,2) node[opendot]{};
\draw[->] (3.5,1.7)--(3.3,1.95) node[pos=0,yshift=-.75em, xshift=3em]{$u_c(c^+)=1$};
\draw[->] (2.5,.3)--(2.8,.05) node[pos=0,yshift=.75em, xshift=-2.85em]{$u_c(c^-)=0$};
}}
\vfill
\[u_c(t) = \begin{cases}  0 & \text{if } t\leq c \\ 1 & \text{if } t> c  \end{cases}\]

}

\slide[Laplace Transform of Heaviside]{
\centerline{\student{\begin{tikzpicture}[xscale=2, yscale=2] \draw[] (0,0)--(0.5,0)--(0.5,0.3)--(1,0.3); \draw[] (-.15,0) node{\tiny 0}; \draw[] (-.15,0.3) node{\tiny 1}; \draw[] (0.5,.45) node{\tiny c};\end{tikzpicture}}}
\student{\algn{\lap{u_c(t)}&=\lapint{u_c(t)}=\int_c^\infty e^{-st}dt=\frac1se^{-sc}\\
&=\boxed{e^{-sc}\frac1s}=e^{-sc}\lap{1}}
\vfill
Q: In general, how can we invert $e^{-sc}\lap{f(t)}$ ?
}
}
\slide[Second Shift Theorem]{
\student{\algn{\lap{f(t-c)u_c(t)}&=\int_0^\infty e^{-st}\underbrace{f(t-c)u_c(t)}_{0 \text{ for } t<c}dt\\
&=\int_c^\infty e^{-st}f(t-c)dt &\arr{l}{u=t-c\\du=dt}\\
&=\int_0^\infty e^{-s(u+c)}f(u)du\\
&= e^{-sc}\int_0^\infty e^{-su}f(u)du=e^{-sc}\lap{f(t)}
}
}
}

\slide{\ex{Suppose $Y(s)=\frac{e^{-s}+e^{-2s} - e^{-3s} - e^{-4s}}{s}$, find and sketch $y(t)$.}
\student{\algn{Y(s)&=\frac{e^{-s}}{s}+\frac{e^{-2s}}{s}-\frac{e^{-3s}}{s}-\frac{e^{-4s}}{s}\\
&=e^{-s}\lap{1} + e^{-2s}\lap{1} -e^{-3s}\lap{1}-e^{-4s}\lap{1}\\
&=u_1(t) \cdot1\evalat{t\to t-c}{} \dots\\
&=u_1(t)  + u_2(t) - u_3(t) -u_4(t)
}\vfill
\tikzplot{.1}{7}{.1}{3}{t}{y(t)}{
\draw[black, ultra thick] (0,0)--(1,0);
\draw[black, dashed, ultra thick] (1,0)--(1,1);
\draw[black, ultra thick] (1,1)--(2,1);
\draw[black, dashed, ultra thick] (2,1)--(2,2);
\draw[black, ultra thick] (2,2)--(3,2);
\draw[black, dashed, ultra thick] (3,2)--(3,1);
\draw[black, ultra thick] (3,1)--(4,1);
\draw[black, dashed, ultra thick] (4,1)--(4,0);
\draw[black, ultra thick] (4,0)--(7,0);
}
}

}

\slide{
\ex{Suppose $Y(s)=e^{-4s} \frac{3}{9+s^2}$, find $y(t)$.}

\student{
\algn{y(t) &= u_4(t) \left[\lapinv{\frac{3}{9+s^2}} \right]_{t=t-4}\\
&= u_4(t) \left[ \sin(3t) \right]_{t=t-4}\\
&= u_4(t)  \sin(3(t-4))}

}
\vfill
\ex{Suppose $Y(s)=e^{-4s} \frac{3}{9+(s+11)^2}$, find $y(t)$.}

\student{
\algn{y(t) &= u_4(t) \left[  e^{-11 t}\lapinv{\frac{3}{9+s^2}} \right]_{t=t-4}\\
&= u_4(t) \left[   e^{-11 t}\sin(3t) \right]_{t=t-4}\\
&= u_4(t)  e^{-11(t-4)}  \sin(3(t-4))}

}

}




\slide[Common Laplace Transforms]{
\vspace{-.5em}
\centerline{$\lap{y\p(t)}=sY(s)-y_0$ \hspace{2em} $\lap{y\pp(t)}=s^2Y(s)-sy_0-v_0$}\vspace{.5em}
\hrule 
\algn{\lap{C}&=\frac{C}{s}&\text{Constant}\\
\lap{t^n}&=\frac{n!}{s^{n+1}}&\text{Power Func.}\\
\lap{e^{at}f(t)}&=F(s-a)&\text{First Shift Theorem}\\
\lap{u_c(t-c)}&=e^{-sc}\frac1s &\text{Heaviside Transfer}\\
\lap{f(t-c)u(t-c)}&=e^{-sc}F(s)&\text{Second Shift Theorem}\\
\lap{t^{n}f(t)}&=(-1)^n\ddn{}{s}{n}F(s)&\text{Resonance}\\
\lap{\sin \omega t}&=\frac{\omega}{\omega^2+s^2}\\
\lap{\cos \omega t}&=\frac{s}{\omega^2+s^2}\\
}
}





\end{document}